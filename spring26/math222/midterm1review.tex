\documentclass[11pt]{exam}
\usepackage[empty]{fullpage}
\usepackage{amsmath}
\usepackage{tikz}
\usepackage{soul}

\newcommand{\ds}{\displaystyle}
\newcommand{\on}{\operatorname}
\newcommand{\Var}{\operatorname{Var}}

\begin{document}
\subsection*{Midterm 1 Review \hfill Math 222}

\textit{These are problems similar to the ones that might be on midterm 1. Be sure to also review class workshops and problems from the textbook.}

\begin{questions}


\question The 2010 General Social Survey asked the question, ``After an average work day, about how many hours do you have to relax or pursue activities that you enjoy?" to a random sample of 1{,}155 Americans. The average relaxing time was found to be 1.65 hours. 

\begin{parts}
\part In this study, identify each of the following: 
\begin{center}
\begin{tabular}{lll}
Observational units: \hspace*{2in} & Variable: \hspace*{2in} \\ \bigskip \\
Parameter: & Statistic: \\
\bigskip
\end{tabular}
\end{center}
\begin{solution}
The observational units are \emph{American adults}. \\
The variable is \emph{hours of relaxation}. \\
The parameter is the \emph{average hours of relaxation in the USA}. \\
The statistic is \emph{1.65 hours}. 
\end{solution}



\end{parts}


\question In order to assess the effectiveness of taking large doses of vitamin C in reducing the duration of the common cold, researchers recruited 400 healthy volunteers from staff and students at a university. A quarter of the patients were assigned a placebo, and the rest were evenly divided between 1g Vitamin C, 3g Vitamin C, or 3g Vitamin C plus additives to be taken at onset of a cold for the following two days. All tablets had identical appearance and packaging. The nurses who handed the prescribed pills to the patients knew which patient received which treatment, but the researchers assessing the patients when they were sick did not. No significant differences were observed in any measure of cold duration or severity between the four medication groups, and the placebo group had the shortest duration of symptoms. 

\begin{parts}
\part Was this an experiment or an observational study? Why?
\begin{solution}
Experiment
\end{solution}
\vfill
\part What are the explanatory and response variables in this study? 
\begin{solution}
Explanatory: Vitamin C. Response: Duration and severity of the cold. 
\end{solution}
\vfill
\part Were the patients blinded to their treatment?
\begin{solution}
Yes.
\end{solution}
\vfill
\part Was this study double-blind? 
\begin{solution}
Yes.
\end{solution}
\vfill
\part Participants are ultimately able to choose whether or not to use the pills prescribed to them. We might expect that not all of them will adhere and take their pills. Does this introduce a confounding variable to the study? Explain your reasoning. 
\begin{solution}
Yes, it could be that the type of pill is associated with whether patients keep using it, for example if it has side effects.  
\end{solution}
\vfill
\end{parts}



\newpage
\question Below are the final exam scores of twenty introductory statistics students.

\begin{verbatim}
    47, 56, 59, 67, 71, 72, 73, 74, 78, 78, 
    79, 79, 81, 81, 82, 83, 83, 88, 89, 94 
\end{verbatim}

\begin{parts}
\part Draw a histogram for the scores. 
\vfill
\vfill

\part Describe the distribution of the data (skewed left, skewed right, or roughly symmetric). 
\begin{solution}
Skewed left.
\end{solution}
\vfill
\part Without calculating anything, which is larger, the mean or the median of this data? 
\begin{solution}
The median will be larger.
\end{solution}
\vfill
\end{parts}



\question Facebook data indicate that 50\% of Facebook users have 100 or more friends, and that the average friend count of users is 190.  What do these findings suggest about the shape of the distribution of number of friends of Facebook users? 
\begin{solution}
Since the average is higher than the median, it suggests that the distribution is skewed right.
\end{solution}
\vfill


\question A food truck sells burgers and fries.  A burger costs \$10, fries cost \$5.  They estimate that 40\% of customers just buy a burger, 50\% buy a burger and fries, and the remaining 10\% just buy fries.  
\begin{parts}
\part Let $X$ represent the amount of money that a random customer will spend at this food truck. Make a table showing the probability model for $X$ (outcomes and probabilities). 
\begin{solution}
\begin{center}
\begin{tabular}{l|c|c|c}
Outcome & \$5 & \$10 & \$15 \\ \hline
Probability & 0.1 & 0.4 & 0.5 
\end{tabular}
\end{center}
\end{solution}
\vfill
\part Find the expected revenue per customer. 
\begin{solution}
\$12.
\end{solution}
\vfill
\part Find the variance and standard deviation in $X$. 
\begin{solution}
The variance is $\sigma^2 = 11$ and the standard deviation is $\$3.3166$.
\end{solution}
\vfill
\part If the food truck has 200 customers in one night, what is the mean and standard deviation in their total revenue for that night? Describe any assumptions you made in order to answer the question.  
\begin{solution}
The mean is \$2400 with a variance of $2200$, so the standard deviation is $\sqrt{2200} = \$46.90$.  You have to assume the amount of money each customer spends is independent of the other customers. 
\end{solution}
\end{parts}

\newpage

\question Men in the United States have an average height of 70.0 inches with a standard deviation 3.0 inches.  Women in the USA have an average height of 64.3 inches with a standard deviation of 2.7 inches.  Assume that both groups have a normal distribution.  

\begin{parts}
\part How tall is a woman in the 90th percentile?  Use the \verb|qnorm(p, mean, sd)| function to express your answer. 
\begin{solution}
$$\mathsf{qnorm}(0.9, 64.3, 2.7)$$
\end{solution}
\vfill

\part What percent of men are between 68 and 72 inches tall? Use the \verb|pnorm(x, mean, sd)| function to express your answer.  
\begin{solution}
$$\mathsf{pnorm}(72, 70, 3) - \mathsf{pnorm}(68, 70, 3)$$
\end{solution}
\vfill


\part What is the probability that a man is over 6 feet tall (72 inches)?  What is the probability that a woman is over 6 feet tall? 
\begin{solution}
$P(\text{Man} > 72) = 1 - \mathsf{pnorm}(72, 70, 3)$ \\
$P(\text{Woman} > 72) = 1 - \mathsf{pnorm}(72, 64.3, 2.7)$.  
\end{solution}
\vfill

\part It turns out that 91\% of women are shorter than 5'8" (68 inches) while only 25\% of men are.  Given that someone is shorter than 5'8", what is the conditional probability that they are a man? Hint: I recommend drawing a tree diagram.
\begin{solution}
$$P(\text{Man} | \text{Height} < 68) = \frac{0.125}{0.125 + 0.455} = 21.55\%.$$
\end{solution}
\vfill
\end{parts}
%\question The scores of high school seniors on the ACT college entrance examination in 2003 had mean $\mu = 20.8$ and standard deviation $\sigma = 4.8$. The distribution of scores is only roughly Normal.
%
%\begin{parts}
%\part If we take a simple random sample of 25 students who took the test, what are the theoretical mean, variance, and standard deviation of the sample mean score $\bar{x}$ of these 25 students? 
%\begin{solution}
%Theoretical mean: $\mu_{\bar{x}} = 20.8$ \\
%Variance: $\sigma_{\bar{x}}^2 = 0.9216$ \\
%Standard deviation: $\sigma_{\bar{x}} = 0.96$ 
%\end{solution}
%\vfill
%\part Estimate the probability $P(\bar{x} > 23)$.  
%\begin{solution}
%\verb|1 - pnorm(23, 20.8, 0.96)|
%\end{solution}
%\vfill
%\part Even though the scores for individuals students are only approximately normally distributed, it is pretty safe to use a normal distribution to answer the question above. Why?  
%\begin{solution}
%Because the distribution gets even more normal when you take a random sample of 25 people. 
%\end{solution}
%\vfill
%\end{parts}


\question When polling companies conduct political polls, they usually only choose samples of between 800 to 1000 people.  What would happen if they choose larger samples?
\begin{choices}
\choice The random error would get larger, but bias would not be a problem.  
\choice The random error would get smaller, but bias would not be a problem.  
\choice The random error would get larger, but bias might still a problem.  
\CorrectChoice The random error would get smaller, but bias might still be a problem.  
\end{choices}


\newpage
\question Probability models are based on assumptions.  Sometimes, we use models even when we know that the assumptions are not all 100\% true. Here are two situations where it is not a good idea to use a given probability model. 

\begin{parts}
\part In one elementary school, there are 56 boys and 44 girls in the 5th grade.  Suppose we randomly assign 25 fifth graders to one teacher's class. Why can't you use the $\on{Binom}(25, 0.56)$ distribution to model the possible number of boys who will end up in that class? 
\begin{solution}
Because the population is only 100 fifth graders, the genders of each student will not be independent.  
\end{solution}
\vfill

\part The National AIDS Behavioral Surveys found that 0.2\% (that's 0.002 as a decimal fraction) of adult heterosexuals had both received a blood transfusion and had a sexual partner from a group at high risk of AIDS. Suppose that this national proportion holds for your region. Explain why you cannot safely use a Normal approximation to model the distribution of the number of people in this group when you interview an random sample of 1000 adults. 
\begin{solution}
Because the expected number of positive responses is only 2 which is not at least 10. 
\end{solution}
\vfill
\end{parts}


\question Suppose $X$ is the result when you roll a six-sided die and $Y$ is the result when you roll a 20-sided die.  The mean and standard deviation for $X$ are $\mu_X = 3.5$, $\sigma_X = 1.7078$.  The mean and standard deviation for $Y$ are $\mu_Y = 10.5$ and $\sigma_Y = 5.7663$.  Compute the following. 
\begin{parts}
\part $E(X + Y)$ 
\begin{solution}
$$E(X+Y) = 14$$
\end{solution}
\vfill
\part $\Var(X + Y)$ 
\begin{solution}
$$\Var(X + Y) = \sigma_X^2 + \sigma_Y^2 = 36.17$$
\end{solution}
\vfill
\part The standard deviation of $X+Y$.  
\begin{solution}
$$\sigma_{X + Y} = \sqrt{\Var(X + Y)} = 6.01$$
\end{solution}
\vfill
\end{parts}

\newpage
\question Data collected at elementary schools in DeKalb County, GA suggest that each year roughly 25\% of students miss exactly one day of school, 15\% miss 2 days, and 28\% miss 3 or more days due to sickness. 

\begin{parts}
\part What is the probability that a student chosen at random doesn't miss any days of school due to sickness this year? 
\begin{solution}
$$32\%$$
\end{solution}
\vfill
\part If a parent has two kids at a DeKalb County elementary school, what is the probability that both of her kids will miss some school?  Note any assumption you make. \vfill
\begin{solution}
If the absences are independent, then 
$$P(X > 0 \text{ and } Y > 0) = 0.68^2 = 46.24\%.$$
\end{solution}
\vfill
\part If a parent has two kids at a DeKalb County elementary school, what is the probability that at least one of her kids will miss some school?  Note any assumption you make.  
\begin{solution}
If the absences are independent, then 
$$P(X > 0 \text{ or } Y > 0) = 0.68 + 0.64 - 0.4624 = 89.76\%.$$
\end{solution}
\vfill
\part Is the assumption that you need to make in parts (b) and (c) reasonable?  Explain. 
\begin{solution}
It might not be reasonable since if one kid gets sick, that increases the odds of the other one getting sick as well.
\end{solution}
\vfill

\end{parts}

\question In the game roulette, if you bet on a color, like black or red, there is an 18/38 chance that you win. If you bet on a number, like 7, there is a 1/38 chance that you win. 
\begin{parts}
\part If you play roulette 100 times, and bet on 7 every time, what is the probability that you win at least 3 times? Use the \verb|pbinom(x, n, p)| function to express your answer.  
\begin{solution}
\verb|1 - pbinom(2, 100, 1/38)|
\end{solution}
\vfill

\part If you play roulette 100 times, and bet on black every time, what is the probability that you win at least 50 times using the normal approximation? Use the \verb|pnorm(x, mean, sd)| function to express your answer.  
\begin{solution}
The mean is $np = 47.37$ and the standard deviation is $\sqrt{np(1-p)} = 4.99$.  So the normal approximation would be:
\verb|1 - pnorm(49.5, 47.37, 4.99)|
\end{solution}
\vfill
\end{parts}

\end{questions}


\end{document}
