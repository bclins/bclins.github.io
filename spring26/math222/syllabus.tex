\documentclass[10pt]{article}

\usepackage[empty]{fullpage}
\usepackage{hyperref}
\usepackage{tikz}
\usepackage{soul}

\begin{document}
\noindent
%
% REQUIRED SYLLABUS CONTENT AS OF SPRING 2025:
% 1. Course Information
%     - Course prefix and number, section number, title, and credit hours
%     - Course description (include full Academic Catalogue description with prerequisites)
%     - Semester and year
%     - Class meeting days, time, and location
%     - Faculty member’s name, contact information, and office hours
%     - Required course materials and texts
%     - Assignment descriptions and calculation of grades
%     - Course schedule
%     - Policies governing late work, make-up assignments, and attendance
% 2. Student Learning Information
%     - Course’s relationship to degree program
%     - Student learning outcomes for major/standalone minor
%     - Course learning outcomes (see template)
% 3. All syllabi must contain the following institutional policies. The wording for these policies should be copied and pasted from the syllabus template.
%     - Grading Scale
%     - Honor Code
%     - Policies for use of AI on assignments
%     - Accommodations


\begin{center}
\includegraphics[scale=0.6]{../../../HSC.png} 
\bigskip

\textbf{Math 222: Statistical Methods (3 credits)} \\
Spring 2026
\end{center}

\noindent
\begin{tabular}{|l|l|}
\hline
Instructor & Brian Lins \\ \hline
Email Address & blins@hsc.edu \\ \hline
Course Meeting Time & MWF 9:30 - 10:20am \\ \hline
Course Meeting Location & Pauley 100  \\ \hline
Office Hours & MWF 10:30 - 11:30am \& W 2:30 - 4:00pm \\ 
& See the course website: \url{https://bclins.github.io} ~ \\ \hline
Textbook & OpenIntro Statistics, 4th edition by Diez,\\ &  Cetinkaya-Rundel, \& Barr \\ \hline
\end{tabular}

\subsubsection*{Course Description}

A project-based study of sampling distributions, estimation, and hypothesis testing. Major topics are
classical and nonparametric analysis of variance, and regression analysis. Students use a variety of
statistical software to produce both individual and group projects. Prerequisite: Mathematics 121, or
consent of the instructor. Offered: spring semester

\subsubsection*{Course Learning Objectives}

\begin{itemize}
\item Students will use RStudio to perform statistical analysis and write reports.
\item Students will be able to set up, calculate, and interpret confidence intervals and hypothesis tests in a variety of situations.
\item Students will be able to implement resampling and nonparameteric methods of data analysis.
\item Students will be able to implement and explain linear, multilinear, and logistic regression.
\end{itemize}

\subsubsection*{Required Materials}

None. See the course website for links to the free textbook.

%\subsubsection*{Assignment Descriptions and Grade}
%
\subsubsection*{Attendance Policy}

Attendance in this class is required. Repeated absences may result in a forced withdrawal from the course. You are responsible for any material you miss due to absence. Please let me know ahead of time if you know that you will not be able to attend class.

\subsubsection*{Grading Policy}

The term grade will be based on the following factors.

\begin{center}
\begin{tabular}{|l|c|}
\hline
Component      & Proportion \\ \hline
Projects & 25\% \\
Midterms  & 45\% \\
Final Exam  & 30\% \\ \hline
\end{tabular}
\end{center}

\subsubsection*{Projects}

There will be homework projects due about once every other week. These projects will require the use of the R programming language. Solutions will be graded on a holistic 4 point scale, with a 4 roughly corresponding to an A, 3 to a B, 2 to a C, 1 to a D, and 0 to an F. Here are the criteria that will be used to judge problems.

\begin{itemize}
\item \textbf{Score 4:} This solution is exceptional. The math is correct, the writing is clear, and the analysis goes beyond the basics and includes additional details beyond the minimum described in the problem. All necessary figures and diagrams have been included, and are well laid out and explained by the surrounding text.
\item \textbf{Score 3:} This solution is good. The math is mostly correct, the writing is clear, and the analysis covers all of the main points that need to be addressed. All important diagrams are included and clear.
\item \textbf{Score 2:} This solution is on the right track. There may be some mathematical mistakes, or the writing is not as clear as it should be. Most of the necessary figures are present, but the layout or explanation may be a little off.
\item \textbf{Score 1:} This solution is deficient. The math may be completely wrong, or there may be serious errors present in the exposition. Figures used to support the solution may be flawed or missing.
\item \textbf{Score 0:} There is no solution. This is the default grade for any problem that has not been submitted.
\end{itemize}


\subsubsection*{Exams}

There will be three in-class midterm exams and a cumulative final. These exams will be announced in advance, and you will know exactly what concepts will be covered on each exam.  

\subsubsection*{Course Schedule} 

The schedule below is tentative, and may be subject to change. Changes will be announced in class, and you are responsible for knowing about any changes even if you miss the class when they are announced. 

\begin{center}
\begin{tabular}{|c|l|l|}
\hline
Week  & Topic & Projects \\ \hline
1 & Intro to R, sampling and experiments & \\
2 & Exploratory data analysis  & \\
3 & Probability and random variables  & Project 1\\
4 & Probability distributions  & \\
5 & Inference for proportions, \textbf{Midterm 1}  & \\
6 & Comparing proportions  &  Project 2 \\
7 & Inference about means  & \\
8 & Analysis of variance   & Project 3 \\
9 & More ANOVA, \textbf{Midterm 2}  & \\
10 & Least squares regression  & Project 4 \\
11 & Logistic regression  & \\
12 & Simulation methods  & Project 5 \\
13 & Bayesian statistics, \textbf{Midterm 3} &  \\
14 & Bayesian methods  & Project 6 \\ \hline
\end{tabular}
\end{center}

\subsubsection*{Late Work Policy}

Please let me know in advance if you will be missing class or if you will be unable to turn in a project on time. Late projects may be accepted on a case-by-case basis.

\subsubsection*{Grading Scale} 

This course adheres to the grades and quality points described in the \href{https://www.hsc.edu/academic-catalogues}{Academic Catalogue}. Consult the Academic Catalogue for a detailed description. 


\subsubsection*{Honor Code}

Students are expected to abide by the Honor Code for all assignments unless a professor indicates otherwise. Students should consult the Academic Catalogue and The Key: The Hampden-Sydney College Student Handbook for the College’s description of the Honor Code and what it identifies as infractions of the Honor Code.

\subsubsection*{Artificial Intelligence Policy}

Artificial intelligence (AI) generators and large language models (LLMs) often rely on existing published materials, and copying or paraphrasing materials generated by AI without attribution is plagiarism. Professors may permit students to use AI generators or LLMs in a variety of ways in their own classes. Those students, however, must not assume that those policies transfer to other classes.

\subsubsection*{Accommodations}

Hampden-Sydney College is committed to ensuring equitable access to its education programs for all students. Under the administration of the Department of Culture and Community, the Office of Accessibility Services (OAS) coordinates reasonable accommodations for qualified students with disabilities. If you wish to seek accommodations for this class, please contact Dr. Melissa Wood, Director of Title IX, Access, and Student Advocacy, at 434-223-6061 or at \url{mwood@hsc.edu}. Additional information may be found here: \url{https://www.hsc.edu/academics/academic-services/disability-services}. Appropriate documentation of disability will be required. For students who have an accommodations letter from OAS, it is essential that you correspond with your professor as soon as possible to discuss your accommodation needs for the course so that appropriate arrangements may be made. 
\end{document}


