\documentclass[12pt]{exam}
\usepackage[empty]{fullpage}
\usepackage{amsmath,amssymb}
\usepackage{hyperref}
\usepackage{tikz}
\usepackage{soul}

\begin{document}
\subsection*{Project 2 \hfill Math 222 \\ Exploratory Data Analysis \hfill Due Friday, February 27}

\begin{questions}

%Researchers have conjectured that the use of the words “forbid” and “allow” can affect people’s
%responses to survey questions. Students in an introductory statistics class were randomly assigned to
%answer one of the following questions:

\question The way that questions are worded in a survey might affect how people respond.  For example, students in one introductory statistics class were randomly assigned to answer one of the following questions:

\begin{itemize}
\item Should your college allow speeches on campus that might incite violence?
\item  Should your college forbid speeches on campus that might incite violence?
\end{itemize}

Of the 11 students who received the first question, 8 responded yes. Of the 14 students who received the
second question, 12 said no.
\begin{parts}
\part Identify the observational units and the explanatory and response variables.
\part Is this an observational study or an experiment? If an observational study, suggest a potential
confounding variable. If an experiment, describe whether you think confounding variables have been adequately controlled.
\part Construct a two-way table in R to summarize these results.
\part Construct a segmented bar graph to display these results and comment on the relationship revealed
by this graph.
\part Based on earlier studies, researchers expected people to be less likely to agree to ``forbid" the
speeches, leading to more no responses (and thus appearing to be in favor of having the speeches),
whereas they expected people to be comparatively less likely to agree to ``allow" the speeches. Do these
data provide strong evidence that these students responded more positively toward having such speeches
if their question was phrased in terms of ``forbid" rather than ``allow"? Use R to carry out a test of significance
and explain the decision you would make based on the p-value. Write a paragraph summarizing your
conclusions including whether a cause-and-effect conclusion can be drawn and the population you are
willing to generalize these results to.
\part In a 1976 study, one group of subjects was asked, ``Do you think the United States should forbid
public speeches in favor of communism?", whereas another group was asked, ``Do you think the United
States should allow public speeches in favor of communism?". Of the 409 subjects randomly asked the
``forbid" version, 161 favored the forbidding of communist speeches. Of the 432 subjects asked the
``allow” version, 189 favored allowing the speeches. Construct a segmented bar graph for these data and
comment on whether you believe the p-value for this table will be larger or smaller than that in
(e). Explain your reasoning.
\end{parts}

\question Myopia, or near-sightedness, typically develops during the childhood years. Recent studies have
explored whether there is an association between development of myopia and the use of night-lights
with infants. Quinn, Shin, Maguire, and Stone (1999) examined the type of light children aged 2-16
were exposed to. Between January and June 1998, the parents of 479 children who were seen as
outpatients in a university pediatric ophthalmology clinic completed a questionnaire (children who had
already developed serious eye conditions were excluded). One of the questions asked was ``Under
which lighting condition did/does your child sleep at night?" before the age of 2 years. The following
two-way table classifies the children’s eye condition and whether or not they slept with some kind of
light (e.g., a night light or full room light) or in darkness.
\begin{verbatim}
---- Example R Code -----------
vision <- matrix(c(40, 39, 12, 114, 115, 22, 18, 78, 22), 
    ncol = 3, 
    byrow = TRUE)
colnames(vision) <- c("Dark", "Night-light", "Room-light")
rownames(vision) <- c("Far-sighted", "Normal", "Near-sighted")    
vision <- as.table(vision)
vision
-------------------------------
\end{verbatim}

\begin{center}
\begin{tabular}{l|c|c|c}
& Dark & Night-light & Room-light \\ \hline
Far-sighted & 40 & 39 & 12 \\ \hline
Normal & 114 & 115 & 22 \\ \hline
Near-sighted & 18 & 78 & 22 
\end{tabular}
\end{center}

\begin{parts}
\part Which variable, lighting condition or eye condition, would you consider the explanatory variable in
this study and which the response variable?
\part Use R to make a segmented bar graph to that shows the percent of kids with each eye condition (with separate bars for each lighting level). Write a few words to describe any relationship or trend that you see in the graph.
\part Carry out a $\chi^2$-test for association to see if the relationship between light at night and near-sightedness is statistically significant.  
\part Make a 95\% confidence interval for the difference in the difference in the proportion of near-sighted kids between kids who sleep in darkness versus kids who sleep with some kind of light at night. Write a sentence or two to explain what your confidence interval means.
\end{parts}
 
\end{questions}

\end{document}
