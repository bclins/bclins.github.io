\documentclass[10pt]{article}

\usepackage[empty]{fullpage}
\usepackage{hyperref}
\usepackage{tikz}
\usepackage{soul}

\begin{document}
\noindent
%
% REQUIRED SYLLABUS CONTENT AS OF SPRING 2025:
% 1. Course Information
%     - Course prefix and number, section number, title, and credit hours
%     - Course description (include full Academic Catalogue description with prerequisites)
%     - Semester and year
%     - Class meeting days, time, and location
%     - Faculty member’s name, contact information, and office hours
%     - Required course materials and texts
%     - Assignment descriptions and calculation of grades
%     - Course schedule
%     - Policies governing late work, make-up assignments, and attendance
% 2. Student Learning Information
%     - Course’s relationship to degree program
%     - Student learning outcomes for major/standalone minor
%     - Course learning outcomes (see template)
% 3. All syllabi must contain the following institutional policies. The wording for these policies should be copied and pasted from the syllabus template.
%     - Grading Scale
%     - Honor Code
%     - Policies for use of AI on assignments
%     - Accommodations


\begin{center}
\includegraphics[scale=0.6]{../../../HSC.png} 
\bigskip

\textbf{Math 342: Numerical Analysis (3 credits)} \\
Spring 2026
\end{center}

\noindent
\begin{tabular}{|l|l|}
\hline
Instructor & Brian Lins \\ \hline
Email Address & blins@hsc.edu \\ \hline
Course Meeting Time & MWF 12:30 - 1:20pm  \\ \hline
Course Meeting Location & Pauley 100  \\ \hline
Office Hours & MWF 10:30 - 11:30am \& W 2:30 - 4:00pm \\ 
& See the course website: \url{https://bclins.github.io} ~ \\ \hline
Required Textbook & None \\ \hline
\end{tabular}

\subsubsection*{Course Description}

This course will focus on the big ideas and problems in numerical analysis, including error analysis, root finding methods, systems of equations, matrix decompositions (including LU, QR, and the singular value decomposition), orthogonal functions, numerical integration, and solving differential equations.

\subsubsection*{Course Learning Objectives}

\begin{itemize}

\item Students will be implement a variety of numerical algorithms using a computer.

\item Students will analyze the accuracy and stability of numerical methods. 

\item Students will be able to choose appropriate numerical methods to solve a wide variety of problems.  

\end{itemize}

\subsubsection*{Required Materials}

None.

%\subsubsection*{Assignment Descriptions and Grade}
%
\subsubsection*{Attendance Policy}

Attendance in this class is required. Repeated absences may result in a forced withdrawal from the course. You are responsible for any material you miss due to absence. Please let me know ahead of time if you know that you will not be able to attend class.

\subsubsection*{Grading Policy}

The term grade will be based on the following factors.

\begin{center}
\begin{tabular}{|l|c|}
\hline
Component      & Proportion \\ \hline
Workshops & 40 \%  \\
Midterm 1 & 15 \%  \\
Midterm 2 & 15 \%  \\
Final Exam & 30 \%  \\ \hline
\end{tabular}
\end{center}



\subsubsection*{Workshops}

Most weeks we will have one or more in-class workshops. We will use the Python programming language for many of these workshops, and will use Google Colab for convenient access to Python. If you already have Python installed on your laptop, then you can use that instead of Google Colab. 

We will do (or at least start) each workshops in class. Any in-class work that you do not finish will become homework that you will need to complete before the next class period.

\subsubsection*{Exams}

There will be two in-class midterm exams and a cumulative final. These exams will be announced in advance, and you will know exactly what concepts will be covered on each exam.  

\subsubsection*{Course Schedule} 

The schedule below is tentative, and may be subject to change. Changes will be announced in class, and you are responsible for knowing about any changes even if you miss the class when they are announced. 

\begin{center}
\begin{tabular}{|c|l|l|}
\hline
Week  & Topic \\ \hline
1  &  Floating point numbers                       \\
2  &  Taylor series                                \\    
3  &  Bisection \& Newton's method                  \\    
4  &  Secant method, fixed point iteration               \\
5  &  Systems of nonlinear equations                            \\
6  &  LU decomposition        \textbf{Midterm 1}           \\
7  &  Norms and inner-products                             \\      
8  &  Gram-Schmidt algorithm                                \\
9  &  Least squares problems and orthogonality              \\
10 &  Fourier series                                 \\
11 &  Numerical integration  \\
12 &  Numerical differentiation, \textbf{Midterm 2}  \\
13 &  Eigenvectors \& eigenvalues                    \\
14 &  Singular value decomposition                   \\ \hline
\end{tabular}
\end{center}

%\subsubsection*{Late Work and Make-Up Assignment Policy}
%
%Please let me know in advance if you will be missing class. If you let me know ahead of time that you will be missing class for a school sponsored event, then we can plan an alternative assignment. If you don't let me know until after the fact, then it will be too late.  There are no make-up quizzes.  That's why I drop your lowest quiz grade. 

\subsubsection*{Grading Scale} 

This course adheres to the grades and quality points described in the \href{https://www.hsc.edu/academic-catalogues}{Academic Catalogue}. Consult the Academic Catalogue for a detailed description. 


\subsubsection*{Honor Code}

Students are expected to abide by the Honor Code for all assignments unless a professor indicates otherwise. Students should consult the Academic Catalogue and The Key: The Hampden-Sydney College Student Handbook for the College’s description of the Honor Code and what it identifies as infractions of the Honor Code.

\subsubsection*{Artificial Intelligence Policy}
 
Artificial intelligence (AI) generators and large language models (LLMs) often rely on existing published materials, and copying or paraphrasing materials generated by AI without attribution is plagiarism. Professors may permit students to use AI generators or LLMs in a variety of ways in their own classes. Those students, however, must not assume that those policies transfer to other classes.

\subsubsection*{Accommodations}

Hampden-Sydney College is committed to ensuring equitable access to its education programs for all students. Under the administration of the Department of Culture and Community, the Office of Accessibility Services (OAS) coordinates reasonable accommodations for qualified students with disabilities. If you wish to seek accommodations for this class, please contact Dr. Melissa Wood, Director of Title IX, Access, and Student Advocacy, at 434-223-6061 or at \url{mwood@hsc.edu}. Additional information may be found here: \url{https://www.hsc.edu/academics/academic-services/disability-services}. Appropriate documentation of disability will be required. For students who have an accommodations letter from OAS, it is essential that you correspond with your professor as soon as possible to discuss your accommodation needs for the course so that appropriate arrangements may be made. 
\end{document}


