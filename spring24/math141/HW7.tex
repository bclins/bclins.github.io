\documentclass[11pt]{article}
\usepackage[margin=0.75in]{geometry}
\usepackage{amsmath}
\usepackage{enumitem}
\usepackage{color,soul}
\usepackage{multicol}
\usepackage{tikz}

\newcommand{\ds}{\displaystyle}

\begin{document}
%%%%%%%%%%%%%%%%%%%%%%%%%%%%%%%%%%%%
%%%
%%% IMPORTANT NOTE TO FUTURE SELF:
%%%
%%% This homework was too hard and too long! Tone it down!  Especially the implicit differentiation where 
%%% they have to separate like terms, those need to be carefully chosen.  Make them easier and the homework will
%%% be much smoother. 
%%%
%%%%%%%%%%%%%%%%%%%%%%%%%%%%%%%%%%%%

\newcounter{enumCount}
\pagestyle{empty}
\subsection*{Math 141 - Homework 7 \hfill Name: \underline{\hspace*{2in}}}

\textit{Find the derivative of each function.}
\begin{multicols}{2}
\begin{enumerate}
\item $f(x) = \sqrt{\dfrac{x - 1}{x+1}}$
\item $y = x \sin \dfrac{1}{x}$
\setcounter{enumCount}{\theenumi}
\end{enumerate}
\end{multicols}
\vfill

\begin{multicols}{2}
\begin{enumerate}
\setcounter{enumi}{\theenumCount}
\item $y = \sqrt{1+\sqrt{1+x}}$
\item $\sin(\sin(\sin x))$
\setcounter{enumCount}{\theenumi}
\end{enumerate}
\end{multicols}
\vfill


\begin{enumerate}
\setcounter{enumi}{\theenumCount}
\item Suppose that the average price of a house in one region is currently $h(t) = \$400,000$ and going up at a rate of \$20,000 per year.   
\begin{enumerate}

\item From the description above, what is $\dfrac{dh}{dt}$ and what are its units?
\vfill

\item An economist estimates that the population of homeless people will increase by about 0.1 for every dollar higher average home prices get. In other words, $\dfrac{dP}{dh} = 0.1$. Use this to estimate the current rate of change in the homeless population per year.  
\vfill

\end{enumerate}
%\item Air is being pumped into a spherical balloon and both the volume $V(t)$ and $r(t)$ can be thought of as functions of time.  If the radius is increasing 

\setcounter{enumCount}{\theenumi}
\end{enumerate}

\noindent
\textit{Use implicit differentiation to find $\dfrac{dy}{dx}$ for each of the following equations.}
\noindent
\begin{multicols}{2}
\begin{enumerate}
\setcounter{enumi}{\theenumCount}
\item $y^2 - x^2 = 9$
\item $x^2 y = y - 5$
\setcounter{enumCount}{\theenumi}
\end{enumerate}
\end{multicols}
\vfill

\noindent
\begin{multicols}{2}
\begin{enumerate}
\setcounter{enumi}{\theenumCount}
\item $x y - \sin(xy) = 1$  % This one was too hard 
\item $\sqrt{xy} = 1 + y^2$  % This could be easier too.
\setcounter{enumCount}{\theenumi}
\end{enumerate}
\end{multicols}
\vfill

\newpage
\noindent
\begin{multicols}{2}
\begin{enumerate}
\setcounter{enumi}{\theenumCount}
\item $\tan(y) = x$
\item $\dfrac{1}{y} = x + \dfrac{1}{x}$
\setcounter{enumCount}{\theenumi}
\end{enumerate}
\end{multicols}
\vfill


\noindent
\textit{Use implicit differentiation to find the equation of the tangent line to the curve at the indicated point.  Then draw and label a graph showing both the curve and the tangent line.  Hint: You can use Desmos to help.}
\begin{enumerate}
\setcounter{enumi}{\theenumCount}
\item (Ellipse) $x^2 + 2xy + 3y^2 = 6$ at $(1,1)$    
\vfill

\item (Tschirnhausen Cubic) $y^2 = x^3 + 3x^2$ at $(1,2)$. 
\vfill

\item Find all points on the ellipse $x^2 - xy + y^2 = 3$ where the tangent line is horizontal. 
\vfill
\setcounter{enumCount}{\theenumi}
\end{enumerate}

\end{document}
