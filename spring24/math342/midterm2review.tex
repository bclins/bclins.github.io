\documentclass[11pt]{exam}
\usepackage[empty]{fullpage}
\usepackage{amsmath,amssymb}
\usepackage{colortbl}
\usepackage{subfig}
\usepackage{multicol}
\usepackage{tikz,pgfplots}
\usepgfplotslibrary{statistics}
\usepackage{color,comment,soul}

\newcommand{\blank}[1]{\underline{\hspace*{#1}}}
\newcommand{\ds}{\displaystyle}
\newcommand{\inner}[1]{\langle {#1} \rangle}


\begin{document}
\graphicspath{{/home/brian/Dropbox/HSC/Spring16/Math111/}}

\subsection*{Numerical Analysis - Math 342 \hfill Midterm 2 Review}

\textit{The following problems are similar to ones you might see on the midterm exam.}

\begin{questions}

\question Use the method of divided differences to find the Newton basis interpolating polynomial for the points $(0,0)$, $(1,1)$, and $(4,2)$.   

\begin{solution}
The table of divided differences is 

\begin{tabular}{c|ccc}
x & y & DD1 & DD2 \\ \hline
0 & 0 & ~  & ~ \\
~ & ~ & 1  & ~ \\
1 & 1 & ~  & $-\tfrac{1}{6}$ \\
~ & ~ & $\tfrac{1}{3}$  & ~ \\
4 & 2 & ~  & ~ \\
\end{tabular}

\noindent
So the interpolating polynomial is $p_2(x) = x - \tfrac{1}{6}x(x-1)$ in the Newton basis.
\end{solution}
\vfill

\question What is the interpolating polynomial above written in terms of the Lagrange basis polynomials? 
\begin{solution}
The Lagrange basis polynomials are
$$L_0(x) = \tfrac{1}{4}(x-1)(x-4) ~~~~ L_1(x) = -\tfrac{1}{3} x(x-4) ~~~~ L_2(x) = \tfrac{1}{12} x(x-1)$$
so the interpolating polynomial is $p_2(x) = -\tfrac{1}{3} x(x-4) + \tfrac{1}{6}x(x-1)$ in the Lagrange basis.
\end{solution}
\vfill

\question Write down the Vandermonde matrix system for these same points $(0,0)$, $(1,1)$, $(4,2)$ to find the coefficients of the interpolating polynomial in the standard basis. You don't need to solve the Vandermonde matrix system.
\begin{solution}
$$\begin{bmatrix} 1 & 0 & 0 \\ 1 & 1 & 1 \\ 1 & 4 & 16\end{bmatrix} \begin{bmatrix} c_0 \\ c_1 \\ c_2 \end{bmatrix} = \begin{bmatrix} 0 \\ 1 \\ 2 \end{bmatrix}.$$
\end{solution}
\vfill

\newpage


\question The second degree interpolating polynomial for the function $f(x) = 1/x$ on the interval $[1,3]$ with three equally spaced nodes ($x_0 = 1$, $x_1=2$, $x_2 =3$) is 
$$p_2(x) = \tfrac{1}{6}x^2 - x + \tfrac{11}{6}.$$
The error formula for interpolating polynomials with equally spaced nodes is 
$$|f(x)-p_n(x)| \le \frac{h^{n+1}}{4(n+1)} \max_{\xi \in [a,b]} |f^{(n+1)}(\xi)|$$ 
where $h = \tfrac{b-a}{n}$.  Use this formula to find an upper bound on the error in using the polynomial $p_2$ to approximate $f(x) = 1/x$ on the interval $[1,3]$.  
\begin{solution}
In this example $n = 2$ and $h = 1$.  The third derivative of $f$ is $f^{(3)}(x) = -6 x^{-4}$. The largest possible absolute value for this function would occur at the left endpoint when $x = 1$, so the error is bounded by:
$$ \frac{h^{n+1}}{4(n+1)} \max_{\xi \in [a,b]} |f^{(n+1)}(\xi)| \le \frac{1}{12} 6(1^{-4}) = \frac{1}{2}.$$ 
\end{solution}
\vfill



\question The formula for Simpson's method (not composite) on an interval is 
$$\int_a^b f(x) \, dx = \frac{h}{3} \left(f(a) + 4 f(m) + f(b) \right) - \frac{h^5}{90} f^{(4)}(\xi)$$
where $h = \frac{b-a}{2}$ and $m = \frac{a+b}{2}$ and $\xi$ is some value between $a$ and $b$. Use this rule to approximate area under the function $y = e^x$ on the interval $[0,2]$ and estimate the error in the approximation.

\begin{solution}
In this example, $a=0$, $b=2$, $m=1$ and $h=1$.  So the approximate area is $\frac{1}{3}(1+4e+e^2) = 6.421$.  The error in this approximation is $- \frac{h^5}{90} f^{(4)}(\xi) = -\frac{1}{90}e^\xi$.  Since $e^\xi$ is at most $e^2$ on the interval, the absolute value of the error is at most $\frac{e^2}{90} = 0.0821$.  
\end{solution}
\vfill

\newpage

\question To estimate the area under a curve using Gaussian quadrature you need to convert the function to an equivalent integral on the interval $[-1,1]$.  Then you can use Gaussian quadrature with any number of nodes.  The formula for Gaussian quadrature with $n=3$ nodes is
$$\int_{-1}^1 f(x) \, dx \approx \tfrac{5}{9}f\left(-\sqrt{\tfrac{3}{5}}\right) + \tfrac{8}{9} f(0) + \tfrac{5}{9} f\left(\sqrt{\tfrac{3}{5}}\right).$$
Find an integral on $[-1,1]$ that is equal to $\int_0^2 e^x \,dx$ and then use Gaussian quadrature to estimate the value of the integral.  

\begin{solution}
Since both $[0,2]$ and $[-1,1]$ have the same length, you just need to translate the function to the left by one.  You can do this by replacing the function $f(x) = e^x$ with a function $g(x) = e^{x+1}$.  Then 
$$\int_0^2 f(x) \, dx = \int_{-1}^1 g(x) \, dx.$$
Using the function $g$ in the formula for Gaussian quadrature I got an approximate area of 
\begin{align*}
\tfrac{5}{9}g\left(-\sqrt{\tfrac{3}{5}}\right) + \tfrac{8}{9} g(0) + \tfrac{5}{9} g\left(\sqrt{\tfrac{3}{5}}\right) &= \tfrac{5}{9}e^{1 - \sqrt{3/5}}+\tfrac{8}{9} e + \tfrac{5}{9} e^{1+\sqrt{3/5}} = 6.389.
\end{align*}

\end{solution}
\vfill


%\question To approximate the improper integral $\ds \int_1^\infty \frac{\sin^2(x)}{x^2} \, dx$, we could use  $\ds \int_1^N \frac{\sin^2(x)}{x^2} \, dx$ for some large $N$. Find a value of $N$ that is large enough so that the absolute error in the approximation is less than $10^{-12}$. Make sure to show why your $N$ works.  
%\begin{solution}
%The error in the approximation is the remaining part of the integral that isn't included in the approximation:
%$$\int_N^\infty \frac{\sin^2(x)}{x^2} \, dx$$
%It is too hard to directly calculate this. Instead, we'll use the fact that $\sin^2(x)$ is always less than or equal to 1 (and nonnegative).  Then we can estimate
%$$\int_N^\infty \frac{\sin^2(x)}{x^2} \, dx \le \int_N^\infty \frac{1}{x^2} \, dx = \left[ \frac{x^{-1}}{-1} \right]^\infty_N = \frac{1}{N}$$
%So we just need $N$ to be at least $10^{12}$ in order for this expression to be smaller than $10^{-12}$. 
%\end{solution}
%\vfill

\question Use the centered difference quotient 
$$\dfrac{f(x+h) - f(x-h)}{2h}$$
to approximate the derivative of $\tan x$ at $x = \pi/3$ with $h = 10^{-5}$.  What is the relative error in your approximation?  
\begin{solution}
The approximation is 
$$\frac{\tan(\tfrac{\pi}{3} + 10^{-5}) - \tan(\tfrac{\pi}{3} - 10^{-5})}{2 \cdot 10^{-5}} = 4.0000000013584724.$$
The exact value of the derivative is
$$\sec^2(\pi/3) = 4.$$
So the relative error is 
$$\frac{|4.0000000013584724 - 4|}{4} = 3.396 \times 10^{-10}.$$
\end{solution}
\vfill



\newpage
\question The normal equation to find the coefficients of a (discrete) least square regression model is 
$$X^T X b = X^T y.$$
Suppose you want the best fit linear function $\hat{y} = b_0 + b_1 x$ to approximate the points $(-2,3)$, $(0, 2)$, $(2,0)$.  
\begin{parts}
\part What is the matrix $X$ and the vector $y$ in the normal equation above?

\begin{solution}
$$X = \begin{bmatrix} 1 & -2 \\ 1 & 0 \\ 1 & 2 \end{bmatrix} ~~~ \text{ and } ~~~ y = \begin{bmatrix} 3 \\ 2\\ 0 \end{bmatrix}.$$
\end{solution}
\vfill

\part Compute $X^TX$ and $X^T y$.  
\begin{solution}
$$X^TX = \begin{bmatrix} 3 &  0 \\ 0 & 8  \end{bmatrix} ~~~ \text{ and } ~~~ X^Ty = \begin{bmatrix} 5 \\ -6 \end{bmatrix}.$$
\end{solution}
\vfill

\part Solve the normal equations to find the slope and y-intercept of the regression line $\hat{y} = b_0 + b_1 x$.  

\begin{solution}
The slope is $b_1 = -3/4$ and the y-intercept is $b_0 = 5/3$.  
\end{solution}
\vfill

\end{parts}

\newpage
\question The Legendre polynomials are a family of orthogonal functions on the interval $[-1,1]$.  The first three Legendre polynomials are 
$$P_0(x) = 1 ~~~~ P_1(x) = x ~~~~ P_2(x) = \tfrac{1}{2}(3x^2 - 1).$$
Using the Legendre functions as a basis, find the best fit (continuous least squares) 2nd degree polynomial approximation of the function $\cos x$ on the interval $[-1,1]$. You can use the following integrals to help
$$\int_{-1}^1 P_0(x) \cos x \,dx = 1.683 ~~~~ \int_{-1}^1 P_1(x) \cos x \,dx = 0$$
and
$$\int_{-1}^1 P_2(x) \cos x \,dx = -0.124 ~~~~ \int_{-1}^1 P_2(x)^2 \, \, dx = 0.4.$$ 
\begin{solution}
The least squares solution is 
$$\frac{\inner{P_0,\cos x}}{\inner{P_0,P_0}} P_0(x) + \frac{\inner{P_1,\cos x}}{\inner{P_1,P_1}} P_1(x) +\frac{\inner{P_2,\cos x}}{\inner{P_2,P_2}} P_2(x)$$
Four of those inner-products were given as integrals above.  The only two left to calculate are:
$$\inner{P_0,P_0} = \int_{-1}^1 1 \,dx = 2$$
and
$$\inner{P_1,P_1} = \int_{-1}^1 x^2 \,dx = \frac{2}{3}$$
So the final answer is 
$$\frac{1.683}{2} P_0(x) - \frac{0.124}{0.4} P_2(x) = 0.8415 - 0.31 \left(\tfrac{1}{2}(3x^2-1)\right).$$
\end{solution}
\vfill



\end{questions}

\end{document}
