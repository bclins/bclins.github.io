\documentclass[12pt]{article}
\usepackage[empty]{fullpage}
\usepackage{amsmath}
\usepackage{tikz}
\usepackage{hyperref}

\newcommand{\blank}[1]{\underline{\hspace*{#1}}}
\newcommand{\ds}{\displaystyle}
\newcommand{\mymod}{\, \mathrm{mod}\,}

\begin{document}

\subsubsection*{Project 1 \hfill Math 121}

\textit{Please type your solutions to the questions below in Microsoft Word or Google Docs (or other document editor) and turn in your solutions in class on \textbf{Friday, November 8}. Use complete sentences to write your answers.  It is okay to discuss the problems with other students, but all of your solutions must be explained in your own words. }  \\


A study published in \emph{Nature} in 2014 reported on an experiment where 38 rhesus monkeys were placed on a restricted calorie diet while another 38 rhesus monkeys were given a normal diet.  Below is a two-way table that shows how many of the monkeys in each group had died by age 33.

\begin{center}
\begin{tabular}{l|c|c}
~ & Restricted calorie diet & Regular diet \\ \hline
Died by age 33 & 26 & 32 \\
Lived past 33 years & 12 & 6 \\ 
\end{tabular}
\end{center}

\begin{enumerate}

\item Use a spreadsheet (e.g., Google Sheets or Microsoft Excel) to make a stacked bar graph that shows the column proportions in the table above (in particular, we want to compare the percent of monkeys that lived longer than 33 years in the two columns).
Cut and paste your graph into your document. Make sure that your graph is clearly labeled.

\item What is the relative risk here? That is, how many times more likely are monkeys to survive if they placed on a restricted calorie diet? 

\item From the data, it looks restricted calorie diets help monkeys live longer, but is this result statistically significant? Carry out an appropriate hypothesis test and explain what the results mean.  Be sure to include a statement of the hypotheses, the test statistic ($z$-value), and the p-value as part of your answer.  

\item Find a 95\% confidence interval for the difference in survival rates for the two groups of monkeys. Explain in words you are 95\% confident is in the interval. 

\item Are the samples sizes big enough to trust the results of the two sample inference techniques used here?  Explain how you can tell.  

\item This was a randomized controlled experiment. Do we still need to worry about lurking variables that might be associated with both survival to age 33 and the type of diet the monkeys got? Explain why or why not.  

\end{enumerate}
\end{document}
