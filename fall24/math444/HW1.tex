\documentclass[11pt]{article}
\usepackage[margin=0.75in]{geometry}
\usepackage{amsmath,amsfonts}
\usepackage{enumitem}
\usepackage{color,soul}

\usepackage{multicol}

\newcommand{\ds}{\displaystyle}
\newcommand{\C}{\mathbb{C}}
\newcommand{\Z}{\mathbb{Z}}
\newcommand{\on}{\operatorname}
\newcommand{\Log}{\on{Log}}
\newcommand{\re}{\operatorname{Re}}
\newcommand{\im}{\operatorname{Im}}


\begin{document}
\newcounter{enumCount}
\pagestyle{empty}
\subsection*{Math 444 - Homework 1 \hfill Name: \underline{\hspace*{2in}}}
\textit{Simplify each of the following expressions as much as you can. Show your work. No calculators.} 
\begin{multicols}{3}
\begin{enumerate}
\item $i^{14}$
\item $(5i)(-2i)(3i)$ 
\item $(3+i)^2$
\setcounter{enumCount}{\theenumi}
\end{enumerate}
\end{multicols}
\vfill

\begin{multicols}{3}
\begin{enumerate}
\setcounter{enumi}{\theenumCount}
\item $\im \left(\dfrac{12}{5 - i}\right)$
\item $(3-2i)(4+i)$
\item $\dfrac{1-i}{1+i}$
\setcounter{enumCount}{\theenumi}
\end{enumerate}
\end{multicols}
\vfill

\begin{multicols}{3}
\begin{enumerate}
\setcounter{enumi}{\theenumCount}
\item $ \left|\dfrac{1}{5 + 12i}\right|$
\item $\overline{(3+4i)(1-i)}$
\item $\overline{e^{i\frac{\pi}{3}}}$
\setcounter{enumCount}{\theenumi}
\end{enumerate}
\end{multicols}
\vfill

\noindent
\textit{Convert the following from rectangular to polar form.}
\begin{multicols}{3}
\begin{enumerate}
\setcounter{enumi}{\theenumCount}
\item $\tfrac{1}{2} + \tfrac{\sqrt{3}}{2}i$
\item $i-1$
\item $\dfrac{i}{1+i}$
\setcounter{enumCount}{\theenumi}
\end{enumerate}
\end{multicols}
\vfill


\noindent
\textit{Convert the following from polar to rectangular form.}
\begin{multicols}{3}
\begin{enumerate}
\setcounter{enumi}{\theenumCount}
\item $e^{5\pi i/3}$
\item $e^{-\pi i /4}$
\item $(\sqrt{3}\, e^{7\pi i /12}) (\sqrt{12}\, e^{29 \pi i/ 12})$
\setcounter{enumCount}{\theenumi}
\end{enumerate}
\end{multicols}
\vfill

\noindent
\textit{Convert to polar or rectangular form to evaluate the following.}
\begin{multicols}{4}
\begin{enumerate}
\setcounter{enumi}{\theenumCount}
\item $\sqrt{2i}$
\item $i^i$
\item $\re \left( 2e^{\pi i/6}\right)$
\item $(i-1)^6$
\setcounter{enumCount}{\theenumi}
\end{enumerate}
\end{multicols}
\vfill

%\begin{multicols}{2}
%\begin{enumerate}
%\setcounter{enumi}{\theenumCount}
%%\item $|1 - e^{i \frac{\pi}{4}}|$
%\setcounter{enumCount}{\theenumi}
%\end{enumerate}
%\end{multicols}
%\vfill


\newpage
\begin{enumerate}
\setcounter{enumi}{\theenumCount}
\item We are going to find the roots of the polynomial equation $z^2 + 2z + (1-i) = 0$ two ways.
\begin{enumerate}
\item Re-write the equation as $z^2 + 2z + 1 = i$ and factor the left hand side (which is a perfect square). Then take the square root of both sides. Remember that all non-zero complex numbers have two square-roots!
\vfill

\item Now use the quadratic formula $\ds z = \frac{-b \pm \sqrt{b^2 - 4ac}}{2a}$.  Do you get the same answer as before? 
\vfill
\end{enumerate}

\item An \textbf{$n$-th root of unity} is a number $z$ such that $z^n = 1$.  Prove that the $n$-th roots of unity are the set $\{e^{2\pi i \frac{k}{n}} : k \in \Z\}$. 
\vfill

\item Find all of the 4th roots of unity. How many are there? Express them in rectangular form.  
\vfill

\item If $z \in \C$ is a root of a polynomial $p$ with real number coefficients, then $\overline{z}$ is also a root of that polynomial because $p(\overline{z}) = \overline{p(z)}$. Find an example to show that this is not true for all polynomials with complex number coefficients.
\vfill

\item Prove that for every $z \in \C$, $|z|^2 = z \overline{z}$.  
\vfill


\item Prove that $|z| = 1$ if and only if $\overline{z} = \dfrac{1}{z}$. % Hint: Use the fact that $|z|^2 = z \cdot \overline{z}$ for every $z \in \C$. 
\vfill

%\item Suppose that $w = r e^{i \omega}$. Find a formula for all $n$ roots of $z^n - w = 0$.  
%\vfill 

%\item Let $z = 1 + \frac{i}{100}$. Find formulas for the real and imaginary parts of $z^n$ (for any integer $n$) that don't involve any complex numbers. 



\setcounter{enumCount}{\theenumi}
\end{enumerate}

\end{document}
