\documentclass[11pt]{article}
\usepackage[margin=0.75in]{geometry}
\usepackage{amsmath,amsfonts}
\usepackage{enumitem}
\usepackage{tikz}
\usepackage{soul}

\usepackage{multicol}

\newcommand{\ds}{\displaystyle}
\newcommand{\N}{\mathbb{N}}
\newcommand{\R}{\mathbb{R}}
\newcommand{\C}{\mathbb{C}}
\newcommand{\re}{\operatorname{Re}}
\newcommand{\im}{\operatorname{Im}}
\newcommand{\Log}{\operatorname{Log}}


\begin{document}
\newcounter{enumCount}
\pagestyle{empty}
\subsection*{Math 444 - Homework 6 \hfill Name: \underline{\hspace*{2in}}}
\begin{enumerate}
\item Let $g(z) = \dfrac{z - i}{2iz + 4}$. Find
\begin{enumerate}
\item $g(i)$ \\
\item $g(0)$ \\
\item $g(\infty)$ \\
\item $g(2i)$ \\
\end{enumerate}


\item Show that the M\"obius transformation $f(z) = \dfrac{1+z}{1-z}$ maps the unit circle (minus the point $z=1$) onto the imaginary axis. Hint: if you transform any three points on a circle, that will determine which circle or line you get.
\vfill



\item Construct a M\"obius transform $f(z)$ that sends $-1$ to infinity, but 0 and 1 are fixed points (i.e., $f(0) = 0$ and $f(1) = 1$).  Hint: First find a M\"obius transform that sends $-1$ to infinity and $0$ to $0$, then rotate/scale until $1$ maps to $1$.  
\vfill

\item Draw a picture to show how the M\"obius transform above (in Problem 3) will transform this shape below (where $T$ is the point $-1$ and $O$ is the origin.  Which circles become lines, and which stay circles? 
\begin{flushright}
\begin{tikzpicture}
\draw (0,0) node {\includegraphics[scale=0.5]{/home/brian/Dropbox/HSC/Spring21/Math444/Website/sangaku.png}};
\end{tikzpicture}
\end{flushright}
%\item Any three distinct points $z_1, z_2, z_3$ uniquely determine either a line or a circle.  You can use M\"obius transformations to find a formula for that line or circle.  The equation
%$$\chi(z) = \frac{(z-z_1)(z_2-z_3)}{(z-z_3)(z_2-z_1)}$$
%is a M\"obius transform that sends $z_1$ to 0, $z_2$ to 1, and $z_3$ to $\infty$.  So it transforms the circle or line that contains $z_1$, $z_2$, and $z_3$ into the line $\R$.  If you calculate the inverse $\chi^{-1}$, then $\chi^{-1}(t)$ will parametrize the line or circle containing $z_1, z_2$, and $z_3$. 
\setcounter{enumCount}{\theenumi}
\end{enumerate}

\newpage 
\noindent
The complex sine and cosine are defined by the formulas
$$\cos z = \frac{e^{iz} + e^{-iz}}{2} \text{ and } \sin z = \frac{e^{iz} - e^{-iz}}{2i}.$$ 

\begin{enumerate}
\setcounter{enumi}{\theenumCount}
\item Show that these formulas agree with the usual sine and cosine when $z$ is a real number. 
\vfill

\item When $z$ is a real number, $\cos z$ and $\sin z$ are always bounded between $1$ and $-1$.  This isn't true for complex numbers.  Find a formula for $\sin(iy)$ for any real number $y$ and then show that  
$$\ds \lim_{y \rightarrow \infty} \sin (i y) = \infty.$$
\vfill

\item Find all solutions of the equation $\sin z = 2$. Hint: Let $u = e^{iz}$.  Then $\sin z = \dfrac{u-u^{-1}}{2i} = 2$.  If you multiply both sides of this equation by $2iu$, then you get a quadratic polynomial, which has two solutions for $u$.  Each of those correspond to a set of solutions for $z$. % NOTE TO SELF: This one was pretty tough, but maybe worthwhile? Might be worth breaking into multiple steps though.  
\vfill

\setcounter{enumCount}{\theenumi}
\end{enumerate}

\noindent
\textit{Convert the following to rectangular form.}
\begin{multicols}{3}
\begin{enumerate}
\setcounter{enumi}{\theenumCount}
\item $e^{\sin (i)}$
\item $\Log(1+\sqrt{3}i)$
\item $\Log\left(\dfrac{1}{3+4i}\right)$
\setcounter{enumCount}{\theenumi}
\end{enumerate}
\end{multicols}
\vfill


\end{document}
