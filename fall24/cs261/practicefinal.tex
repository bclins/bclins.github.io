\documentclass[12pt]{exam}
\usepackage{amsmath,amssymb}
\usepackage[margin=0.5in]{geometry}
\usepackage{tikz}
\usepackage{soul}

\newcommand{\blank}[1]{\underline{\hspace*{#1}}}
\newcommand{\ds}{\displaystyle}
\newcommand{\on}{\operatorname}


\begin{document}
\pagestyle{empty}
\graphicspath{{/home/brian/Dropbox/HSC/Spring16/Math111/}}

\subsection*{Practice Final Exam - CS 261}
\textit{Here are questions similar to what will be on the final exam.  You won't be able to use any outside material during the final exam (no notes, computers, etc.) so see how much you can answer without looking things up. Be sure to study any Python rules and terminology that you don't know. }

\begin{questions}

\question What is the difference between \verb|"True"| and \verb|True| in Python? 
\begin{solution}
One is a string while the other is a boolean. 
\end{solution}
\vfill

\question Name three different sequence types. Describe how they differ.
\vfill

\question Which of the following is a list method?
\begin{choices}
\choice \verb|.items()|
\choice \verb|.keys()|
\choice \verb|len()|
\CorrectChoice \verb|.append()|
\choice \verb|.length()|
\end{choices}
\smallskip

\question What is the difference between a for-loop and a while-loop?  
\vfill

\newpage
\question Write a function \verb|near_and_far()| that accepts three ints \verb|a|, \verb|b|, and \verb|c| as arguments. Return \verb|True| if one of \verb|b| or \verb|c| is ``close" to \verb|a| (differing from \verb|a| by at most 1), while the other is ``far", differing from \verb|a| by 2 or more. Hint: you can use the \verb|abs()| function to find the absolute value of any number.  
\vfill

\question Consider the following recursive function 

\begin{verbatim}
def next_num(n):
    if n == 1:
        return 2
    elif n == 2:
        return 3
    else:
        return next_num(n-1) * next_num(n-2)
\end{verbatim}

Calculate the values \verb`next_num(3)`, \verb`next_num(4)`, and \verb`next_num(5)`.  
\vfill

\newpage

\question Write a function called \verb|separate(nums)| that inputs a list of integers and returns a tuple containing two lists, one containing the even elements in \verb|nums| and the other containing the odd elements.  
\vfill

\question Suppose that Bob writes a program called \verb|bob.py| that contains the following code:
\begin{verbatim}
-- bob.py ------------------------
print("What's my name?")

if __name__ == "__main__":
    print("My name is __main__.")
elif __name__ == "bob":
    print("My name is bob.")
else:
    print("I don't know my name.")
----------------------------------
\end{verbatim}

If Alice creates a program called \verb|alice.py| that contains the following single import statement and is in the same folder as \verb|bob.py|, then what will happen when Alice runs her program? 
\begin{verbatim}
-- alice.py ----------------------
import bob
----------------------------------
\end{verbatim}
\bigskip

\newpage
\question Describe at least three differences between dictionaries and lists.  
\vfill

\question What will the following code print? 
\begin{verbatim}
def f(a, b):
    a += 1
    b.append(1)

a = 5
b = []
f(a, b)
print(a)
print(b)
\end{verbatim}
\bigskip
\bigskip

\question Consider the following class.  

\begin{verbatim}
class Mystery:
    def __init__(self, x, y, z):
        self.data = [x, y, z]
    def method1(self):
        self.data[0] += 2
        self.data[1] += 1
    def method2(n):
        return Mystery(n, 2*n, 3*n)
    def __str__(self):
        return f"Mystery({self.data[0]},{self.data[1]},{self.data[2]})"
\end{verbatim} 

\begin{parts}
\part What will the following code print?  
\begin{verbatim}
a = Mystery(1,2,3)
a.method1()
print(a)
\end{verbatim}
\bigskip
\bigskip

\part What will the following code print?
\begin{verbatim}
b = Mystery(4,5,6)
print(b.method1())
\end{verbatim}
\bigskip
\bigskip

\part How would you call \verb|method2| if you wanted \verb|n| to be 5?  
\bigskip
\bigskip

\end{parts}

\end{questions}

\end{document} 
