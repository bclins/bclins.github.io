\documentclass[12pt]{article}
\usepackage{amsmath}
\usepackage{tikz}
\usepackage{soul}
\usepackage[empty]{fullpage}

\begin{document}
\section*{Project 6 \hfill CS 261}

\textit{Save your program for this project as} \verb|<emailID>_project6.py| \textit{where} \verb|<emailID>| \textit{is the part of your Hampden-Sydney e-mail address before the @ symbol. When you are finished, e-mail your program to} \verb|blins@hsc.edu|. \textit{Your solution is due by noon on Friday, October 18. }

\subsection*{Caesar Cipher 2}

For this project you will need to download and save the file 
\begin{center}
\verb|https://bclins.github.io/fall24/cs261/poem.txt|
\end{center}
The contents of this file have been encrypted using a Caesar shift. You'll need to write a program to open the file and read its contents.  Then apply the appropriate Caesar shift to decrypt the file and print the decrypted poem.  When you are finished, submit your Python code. You do not need to submit the decrypted poem, but you do need to include the shift value that worked to decrypt the poem in your submission.  

\subsubsection*{Hints}

\begin{enumerate}
\item If it helps, you can use your code from last week to perform the Caesar shift. 

\item To figure out the correct shift to decrypt the message, you can either focus on short one, two, and common three letter words to make an educated guess.  Or you can use trial and error to find the shift that works.  

\end{enumerate}



\end{document}
