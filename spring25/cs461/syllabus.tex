\documentclass[10pt]{article}

\usepackage[empty]{fullpage}
\usepackage{hyperref}
\usepackage{tikz}
\usepackage{soul}

\begin{document}
%
% REQUIRED SYLLABUS CONTENT AS OF SPRING 2025:
% 1. Course Information
%     - Course prefix and number, section number, title, and credit hours
%     - Course description (include full Academic Catalogue description with prerequisites)
%     - Semester and year
%     - Class meeting days, time, and location
%     - Faculty member’s name, contact information, and office hours
%     - Required course materials and texts
%     - Assignment descriptions and calculation of grades
%     - Course schedule
%     - Policies governing late work, make-up assignments, and attendance
% 2. Student Learning Information
%     - Course’s relationship to degree program
%     - Student learning outcomes for major/standalone minor
%     - Course learning outcomes (see template)
% 3. All syllabi must contain the following institutional policies. The wording for these policies should be copied and pasted from the syllabus template.
%     - Grading Scale
%     - Honor Code
%     - Policies for use of AI on assignments
%     - Accommodations


\begin{center}
\includegraphics[scale=0.6]{../../../HSC.png} 
\bigskip

\textbf{Computer Science 461: Theory of Computing (3 credits)} \\
Spring 2025
\end{center}

\noindent
\begin{tabular}{|l|l|}
\hline
Instructor & Brian Lins \\ \hline
Email Address & blins@hsc.edu \\ \hline
Course Meeting Time & MWF 1:30 - 2:20pm \\ \hline
Course Meeting Location & Pauley 100 \\ \hline
Office Hours & Wednesdays 2:30 - 4:00pm \& Thursdays 1:00 - 2:30pm \\ 
& See the course website: \url{https://bclins.github.io} ~ \\ \hline
\end{tabular}

\subsubsection*{Course Description}

An introduction to theoretical computer science. Abstract models of computers are used to help investigate the limitations of computing. Topics may include computability, complexity, automata, formal languages and grammars, and the Chomsky hierarchy. Prerequisites: Computer Science 262 and Mathematics 254. 

\subsubsection*{Course Learning Objectives}

\begin{itemize}
\item Learn the definitions and limitations of models of computation including finite automata, context-free grammars, and Turing machines.
\item Learn the connection between models of computation and formal languages.  
\item Understand computational complexity classes including P, NP, and NP-complete.
\item Gain experience with creative mathematical problem solving and develop the ability to write correct, clear, and concise mathematical proofs.
\end{itemize}

%Boston University: 
%
%\begin{itemize}
%\item Foremost, understand how to rigorously reason about computation through the use of abstract, formal models.
%\item Learn the definitions of several specific models of computation including finite automata, context-free grammars, and Turing machines, learn tools for analyzing their power and limitations, and understand how they are used in other areas of computer science.
%\item Learn how fundamental philosophical questions about the nature of computation (Are there problems which cannot be solved by computers? Can every problem for which we can quickly verify a solution also be solved efficiently?) can be formalized as precise mathematical problems.
%\item Gain experience with creative mathematical problem solving and develop the ability to write correct, clear, and concise mathematical proofs.
%\end{itemize}
%
%University of Toledo
%
%The students will be able to:
%1. Devise a variety of simple proofs.
%2. Define what a Regular Language is and construct a finite state machine for it.
%3. Construct equivalent representations among Regular Languages, Regular Expressions, and Regular Grammars.
%4. Formulate a grammar defining the syntax of common programming languages.
%5. Be able to formulate the equations for pushdown automaton.
%6. Understand Turing Machines and the simple primitive mechanisms needed for all computation.
%7. Understand recursive and recursively enumerable languages.
%8. Identify the characteristics of problems for which no computational solution exists.
%9. Understand the concepts of P vs. NP vs. NP-complete.

\subsubsection*{Required Materials}

None.  See the course website for links to the free textbook.

%\subsubsection*{Assignment Descriptions and Grade}

\subsubsection*{Attendance Policy}

Attendance in this class is required. Repeated absences may result in a forced withdrawal from the course. You are responsible for any material you miss due to absence. Please let me know ahead of time if you know that you will not be able to attend class.

\subsubsection*{Grading Policy}

The term grade will be based on the following factors.

\begin{center}
\begin{tabular}{|l|c|}
\hline
Component      & Proportion \\ \hline
Homework  & 40\% \\
Midterms  & 30\% \\
Final Exam  & 30\% \\ \hline
\end{tabular}
\end{center}

\subsubsection*{Homework}

There will be homework problems assigned every week. These will typically be collected on Mondays. I will drop the lowest homework grade. Late homework will only receive a fraction of the full possible grade.

\subsubsection*{Exams}

There will be three in-class midterm exams and a cumulative final. These exams will be announced in advance, and you will know exactly what concepts will be covered on each exam.

\subsubsection*{Course Schedule} 

The schedule below is tentative, and may be subject to change. Changes will be announced in class, and you are responsible for knowing about any changes even if you miss the class when they are announced.

\begin{center}
\begin{tabular}{|c|l|}
\hline
Week  & Topic \\ \hline
1  & Boolean circuits  \\
2  & Finite automata  \\
3  & Nondeterministic finite automata  \\
4  & Regular expressions \& languages  \\
5  & Pumping lemma, Midterm 1  \\
6  & Context free languages  \\
7  & Pushdown automata  \\
8  & Pumping lemma 2  \\
9  & Turing machines  \\
10  & Decidability, Midterm 2  \\
11  & Enumerability  \\
12  & Complexity theory  \\
13  & Nodeterministic algorithms  \\
14  & NP-completeness, Midterm 3  \\ \hline
\end{tabular}
\end{center}


\subsubsection*{Late Work and Make-Up Assignment Policy}

Please let me know in advance if you will be missing class. Late homework assignments may be accepted with a grade penalty. 


\subsubsection*{Student Learning Outcomes}

As part of a major offer by the Math \& Computer Science department the SLOs for this course are:
\begin{itemize}
\item \textbf{Problem solving} - Students will be able to apply appropriate principles and techniques to solve problems.
\item \textbf{Rigor} - Students will be able to read and construct mathematical proofs, and to provide counterexamples to false propositions.
\item \textbf{Written communication} - Student will be able to communicate mathematical information in a clear and concise manner in standard written English to both a technical and non-technical audience.
\end{itemize}

\subsubsection*{Grading Scale} 

This course adheres to the grades and quality points described in the \href{https://www.hsc.edu/academic-catalogues}{Academic Catalogue}. Consult the Academic Catalogue for a detailed description. 


\subsubsection*{Honor Code}

Students are expected to abide by the Honor Code for all assignments unless a professor indicates otherwise. Students should consult the Academic Catalogue and The Key: The Hampden-Sydney College Student Handbook for the College’s description of the Honor Code and what it identifies as infractions of the Honor Code.

\subsubsection*{Artificial Intelligence Policy}

Artificial intelligence (AI) generators and large language models (LLMs) often rely on existing published materials, and copying or paraphrasing materials generated by AI without attribution is plagiarism. Professors may permit students to use AI generators or LLMs in a variety of ways in their own classes. Those students, however, must not assume that those policies transfer to other classes.

\subsubsection*{Accommodations}

Hampden-Sydney College is committed to ensuring equitable access to its education programs for all students. Under the administration of Culture and Inclusion, the Office of Accessibility Services (OAS) coordinates reasonable accommodations for qualified students with disabilities. If you wish to seek accommodations for this class, please contact Dr. Melissa Wood, Director of Title IX, Access, and Inclusion, at 434-223-6061 or at \url{mwood@hsc.edu}. Additional information may be found here: \url{https://www.hsc.edu/academics/academic-services/disability-services}. Appropriate documentation of disability will be required. For students who have an accommodations letter from OAS, it is essential that you correspond with your professor as soon as possible to discuss your accommodation needs for the course so that appropriate arrangements may be made.
\end{document}


