\documentclass[12pt]{exam}

\usepackage[margin=0.5in]{geometry}
\usepackage{amsmath,amssymb}
\usepackage{multicol}
\usepackage{tikz,soul}
\usepackage{diagbox}
\usetikzlibrary{arrows,automata,positioning}

\newcommand{\ds}{\displaystyle}
\newcommand{\bs}{\backslash}
\newcommand{\on}{\operatorname}
\newcommand{\R}{\mathbb{R}}
\newcommand{\Z}{\mathbb{Z}}
\newcommand{\N}{\mathbb{N}}

\begin{document}
\pagestyle{empty}
\subsubsection*{Homework 10 - Computer Science 461}
\textit{Due Monday, April 14}

\begin{questions}
\question For each of the following, determine if the statement is true or false.  Briefly explain your answers.
\begin{multicols}{2}
\begin{parts}
\part $n^2 \in O(n)$.\\ \bigskip 
\part $n^2 + 3 \log n \in O(n^2 \log n)$.
\part $n^3 + \sqrt{2^n} \in O(n^3)$.  \\ \bigskip
\part $2^{2^n} \in O(n^n)$. 
\end{parts}
\end{multicols}


\begin{solution}
(a) False, (b) True, (c) False, (d) False since $2^{2^n} = (2^n)^{2^n/n} \gg n^n$. 
\end{solution}
\vspace*{0.5in}


\question Let $L = \{ w \in \{a,b\}^* : w \text{ has an equal number of a's and b's} \}$. Prove that $L \in \mathsf{P}$.  
\begin{solution}
You can prove this by observing that this is a context-free language, so it must be in $\mathsf{P}$. 
\end{solution}
\vfill

\question Suppose that $L, K \subset \{0,1\}^*$ are languages in class $\textsf{P}$.  Prove the following are also in class $\textsf{P}$. 
\begin{parts}
\part The complement of $L$.
\begin{solution}
If there is an algorithm $A$ that decides whether $w \in L$ in polynomial time, then the algorithm that outputs the opposite of $A$'s result also runs in polynomial time. 
\end{solution}
\vfill

\part The union $L \cup K$.
\begin{solution}
If there are algorithms $A_L$ and $A_K$ that decide $L$ and $K$ respectively in polynomial time, then $A_L$ on $w$.  If it accepts, then accept, otherwise run $A_K$ on $w$ and accept accordingly.  At worst, this has running time equal to the sum of the running times of $A_L$ and $A_K$, so that is polynomial. 
\end{solution}
\vfill
\end{parts}


\end{questions}
\end{document}
