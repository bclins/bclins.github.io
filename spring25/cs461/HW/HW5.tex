\documentclass[12pt,answers]{exam}

\usepackage[margin=0.5in]{geometry}
\usepackage{amsmath,amssymb}
\usepackage{tikz,soul}
\usepackage{diagbox}
\usetikzlibrary{arrows,automata,positioning}

\newcommand{\ds}{\displaystyle}
\newcommand{\bs}{\backslash}
\newcommand{\on}{\operatorname}
\newcommand{\R}{\mathbb{R}}
\newcommand{\Z}{\mathbb{Z}}
\newcommand{\N}{\mathbb{N}}

\begin{document}
\pagestyle{empty}
\subsubsection*{Homework 5 - Computer Science 461 \hfill Name: \underline{\hspace*{2in}}}

\textit{Due Monday, Feb 17.} % You can e-mail your code for the computer programming problems to me at }\verb|blins@hsc.edu|.

\begin{questions}


\question Let $L = \{w \in \Sigma^* : \text{ the length of }w\text{ is a power of 2} \}$. Use the pumping lemma to prove that $L$ is not a regular language.  

\vfill

\question Let $L = \{u w u : u, w \in \{0,1\}^*, u \ne \epsilon \}$. Prove that $L$ is not a regular language. 
\vfill

%% I don't know if this next one is even true?
%\question Let $\Sigma = \{0, \ldots, 9\}$ and let $L$ be the set of base-10 representations of powers of 2.  So 
%$$L = \{1, 2, 4, 8, 16, 32, 64, 128, \ldots \}.$$
%Use the pumping lemma to prove that $L$ is not regular. 
%\vfill
%\begin{solution}
%Suppose that $w$ is a string in $L$ long enough to be pumped, i.e., $w = xyz$ where $xy^kz \in L$ for all $k \ge 0$, and $|y| \ge 1$.  At the same time, $w$ represents a number $2^n$ for some $n$.  So $xy^kz$ represents a number given by 
%$$x*10^{k|y|+|z|} + y * (10^{(k-1)|y|+|z|} + 10^{(k-2)|y| + |z|} + \ldots + 10^{|z|}) + z$$
%and the difference between $xy^{k+1}z$ and $xy^kz$ is:
%$$x*(10^{(k+1)|y|+|z|} - 10^{k|y|+|z|}) + y * 10^{k|y|+|z|}= 10^{k|y|+|z|}(x (10^{|y|} - 1) + y)$$
%which grows exponentially in $k$.  But I don't think this equation is always going to work out... you just need to prove that.  Yeah, this does lead to an impossible equation I'm pretty sure based on growth rates, the differences between powers of 2 must grow like powers of 2 but the difference on the other side must grow like powers of 10, but I'm not sure how to make that precise.  
%\end{solution}

\vfill

\end{questions}
\end{document}
