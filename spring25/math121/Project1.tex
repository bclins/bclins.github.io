\documentclass[12pt]{article}
\usepackage[empty]{fullpage}
\usepackage{amsmath}
\usepackage{tikz}
\usepackage{hyperref}

\newcommand{\blank}[1]{\underline{\hspace*{#1}}}
\newcommand{\ds}{\displaystyle}
\newcommand{\mymod}{\, \mathrm{mod}\,}

\begin{document}

\subsubsection*{Project 1 \hfill Math 121}

\textit{Please type your solutions to the questions below in Microsoft Word or Google Docs (or other document editor) and turn in your solutions in class on \textbf{Friday, April 4}. Your grade will be based on three factors: completeness, correctness, and style. To get full style credit you should write all answers in complete sentences. It is okay to discuss the problems with other students, but all of your solutions must be explained in your own words. }  \\


A study published in \emph{Nature} in 2014 reported on a study where 38 rhesus monkeys were placed on a restricted calorie diet while another 38 rhesus monkeys were given a normal diet.  Below is a table that shows how many of the monkeys in each group had died by age 33.

\begin{center}
\begin{tabular}{l|c|c}
~ & Restricted calorie diet & Regular diet \\ \hline
Died by age 33 & 26 & 32 \\
Lived past 33 years & 12 & 6 \\ \hline
Total & 38 & 38
\end{tabular}
\end{center}

\begin{enumerate}


\item Write a short introductory paragraph about this study.  What were the explanatory and response variables that the researchers were interested in?  What were the survival rates (proportion that survived) for each of the two groups of monkeys, and how big was the difference? How did the researchers investigate the relationship between the two variables?  Did they use an experiment or an observational study? 

%\item What is the relative risk here? That is, how many times more likely are monkeys to survive if they placed on a restricted calorie diet? 

\item From the data, it looks restricted calorie diets help monkeys live longer, but is this result statistically significant? Carry out an appropriate hypothesis test and explain what the results mean.  Be sure to include a statement of the hypotheses, the test statistic ($z$-value), and the p-value as part of your answer.  

\item Find a 95\% confidence interval for the difference in survival rates for the two groups of monkeys. Explain in words you are 95\% confident is in the interval. 

\item Are the samples sizes big enough to trust the results of the two sample inference techniques used here?  Explain how you can tell.  


\item Write a concluding paragraph to summarize your results. Should we be worried about lurking variables, and if not, why not?  Should we conclude that low calorie diets cause monkeys to live longer? Explain the reasons for your conclusion.
\end{enumerate}
\end{document}
