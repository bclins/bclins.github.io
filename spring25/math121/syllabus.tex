\documentclass[10pt]{article}

\usepackage[empty]{fullpage}
\usepackage{hyperref}
\usepackage{tikz}
\usepackage{soul}

\begin{document}
\noindent
%
% REQUIRED SYLLABUS CONTENT AS OF SPRING 2025:
% 1. Course Information
%     - Course prefix and number, section number, title, and credit hours
%     - Course description (include full Academic Catalogue description with prerequisites)
%     - Semester and year
%     - Class meeting days, time, and location
%     - Faculty member’s name, contact information, and office hours
%     - Required course materials and texts
%     - Assignment descriptions and calculation of grades
%     - Course schedule
%     - Policies governing late work, make-up assignments, and attendance
% 2. Student Learning Information
%     - Course’s relationship to degree program
%     - Student learning outcomes for major/standalone minor
%     - Course learning outcomes (see template)
% 3. All syllabi must contain the following institutional policies. The wording for these policies should be copied and pasted from the syllabus template.
%     - Grading Scale
%     - Honor Code
%     - Policies for use of AI on assignments
%     - Accommodations


\begin{center}
\includegraphics[scale=0.6]{../../../HSC.png} 
\bigskip

\textbf{Math 121: Statistics (3 credits)} \\
Spring 2025
\end{center}

\noindent
\begin{tabular}{|l|l|}
\hline
Instructor & Brian Lins \\ \hline
Email Address & blins@hsc.edu \\ \hline
Course Meeting Time & MWF 12:30 - 1:20pm \\ \hline
Course Meeting Location & Pauley 105 \\ \hline
Office Hours & Wednesdays 2:30 - 4pm \& Thursdays 12:30 - 2pm \\ 
& See the course website: \url{https://bclins.github.io} ~ \\ \hline
\end{tabular}

\subsubsection*{Course Description}

Introduction to probability and statistics. Exploratory data analysis. Discrete and continuous random variables, estimation, hypothesis testing

\subsubsection*{Course Learning Objectives}

\begin{itemize}

\item Students will be able to read and interpret graphs of quantitative data.
\item Students will be able to perform calculations related to the normal distribution. 
\item Students will be able to set up, calculate, and interpret both confidence intervals and hypothesis tests.

\end{itemize}

\subsubsection*{Required Materials}

None. See the course website for links to the free textbook.

%\subsubsection*{Assignment Descriptions and Grade}
%
\subsubsection*{Attendance Policy}

Attendance in this class is required. Repeated absences may result in a forced withdrawal from the course. You are responsible for any material you miss due to absence. Please let me know ahead of time if you know that you will not be able to attend class.

\subsubsection*{Grading Policy}

The term grade will be based on the following factors.

\begin{center}
\begin{tabular}{|l|c|}
\hline
Component      & Proportion \\ \hline
Midterms  & 45\% \\
Projects  & 10\% \\
Quizzes  & 20\% \\
Final Exam  & 25\% \\ \hline
\end{tabular}
\end{center}

\subsubsection*{In-Class Problems}

There is a saying that, “you learn math by doing math.” This is very true! During most class periods you will be asked to solve problems in-class. Any problems you do not complete during class become homework problems. I do not collect these problems however. Instead, you may consult your notes and completed classwork during the weekly quizzes.


\subsubsection*{Quizzes}

Every Friday (except when there is a midterm exam) there will be a 10-15 minute quiz on recently covered material. Each quiz will have 3 or 4 problems and will cover material similar to the in-class and take-home problems. During quizzes you are allowed to use your notes and solutions to old class workshop problems. The lowest quiz grade will be dropped from the final average.

\subsubsection*{Projects}

There will be four homework projects at the end of the semester. These projects will require you to analyze real world data, make graphs, carry out relevant statistical tests, and explain your conclusions. The projects must by typed using a computer and you can e-mail them to me when they are due. If you wish, you may work with a partner on the project. If you do work with a partner then you can both submit one file together, just be sure that both of your names are clearly listed on the file that you submit. Aside from work with your partner, all work should be your own. Each project will be graded holistically using the following rubric:

\begin{itemize}
\item Grade: A. Every part of the assignment is complete and your work is clear and correct.
\item Grade: B. Almost every part of the assignment is complete and your work is clear. There may be a few minor mistakes, but no major errors.
\item Grade: C. Most of the assignment is complete, but is either not clear or has significant mistakes.
\item Grade: D. You turn in something, but it has major errors or omissions.
\end{itemize}

The dates listed for the projects on the schedule above are when the projects will be assigned, not when they are due. You will have at least 3 days from when a project is assigned until it will be due.

\subsubsection*{Exams}

There will be three in-class midterm exams and a cumulative final. These exams will be announced in advance, and you will know exactly what concepts will be covered on each exam.

\subsubsection*{Course Schedule} 

The schedule below is tentative, and may be subject to change. Changes will be announced in class, and you are responsible for knowing about any changes even if you miss the class when they are announced. 

\begin{center}
\begin{tabular}{|c|l|l|}
\hline
Week  & Topic & Projects \\ \hline
1  & Data tables \& summaries  &  \\
2  & Standard deviation \& normal distribution  &  \\
3  & Normal distribution probabilities  &  \\
4  & Linear regression  &  \\
5  & Sampling, Midterm 1  &  \\
6  & Randomized experiments \& probability  &  \\
7  & Law of large numbers  &  \\
8  & Sampling distributions  &  \\
9  & Confidence intervals, Midterm 2  &  \\
10  & Hypothesis testing  & Project 1 \\
11  & Comparing two proportions  & Project 2 \\
12  & Inference about means  & Project 3 \\
13  & Two sample t-methods, Midterm 3  & Project 4 \\
14  & Chi-squared test for association  &  \\ \hline
\end{tabular}
\end{center}

\subsubsection*{Late Work and Make-Up Assignment Policy}

Please let me know in advance if you will be missing class. If you let me know ahead of time that you will be missing class for a school sponsored event, then we can plan an alternative assignment. If you don't let me know until after the fact, then it will be too late.  There are no make-up quizzes.  That's why I drop your lowest quiz grade. 

\subsubsection*{Grading Scale} 

This course adheres to the grades and quality points described in the \href{https://www.hsc.edu/academic-catalogues}{Academic Catalogue}. Consult the Academic Catalogue for a detailed description. 


\subsubsection*{Honor Code}

Students are expected to abide by the Honor Code for all assignments unless a professor indicates otherwise. Students should consult the Academic Catalogue and The Key: The Hampden-Sydney College Student Handbook for the College’s description of the Honor Code and what it identifies as infractions of the Honor Code.

\subsubsection*{Artificial Intelligence Policy}

Artificial intelligence (AI) generators and large language models (LLMs) often rely on existing published materials, and copying or paraphrasing materials generated by AI without attribution is plagiarism. Professors may permit students to use AI generators or LLMs in a variety of ways in their own classes. Those students, however, must not assume that those policies transfer to other classes.

\subsubsection*{Accommodations}

Hampden-Sydney College is committed to ensuring equitable access to its education programs for all students. Under the administration of Culture and Inclusion, the Office of Accessibility Services (OAS) coordinates reasonable accommodations for qualified students with disabilities. If you wish to seek accommodations for this class, please contact Dr. Melissa Wood, Director of Title IX, Access, and Inclusion, at 434-223-6061 or at \url{mwood@hsc.edu}. Additional information may be found here: \url{https://www.hsc.edu/academics/academic-services/disability-services}. Appropriate documentation of disability will be required. For students who have an accommodations letter from OAS, it is essential that you correspond with your professor as soon as possible to discuss your accommodation needs for the course so that appropriate arrangements may be made.
\end{document}


