\documentclass[10pt]{article}
\usepackage[margin=0.5in]{geometry}
\usepackage{amsmath}
\usepackage{enumitem}
\usepackage{multicol}
\usepackage{tikz}
\usepackage{soul}

\newcommand{\ds}{\displaystyle}
\newcommand{\on}{\operatorname}


\begin{document}
\newcounter{enumCount}
\pagestyle{empty}
\subsection*{Homework 8 - Math 140 \hfill Name: \underline{\hspace*{2in}}}

\noindent
\textit{Calculate the following derivatives.  }


\begin{multicols}{2}
\begin{enumerate}
\item $\ds \frac{d}{dx} \sqrt{x}(x^2 - 4)$
\item $\ds \frac{d}{dx} \left(\frac{3}{x} - \frac{4}{x^2}\right)$
\setcounter{enumCount}{\theenumi}
\end{enumerate}
\end{multicols}
\vfill

\begin{multicols}{2}
\begin{enumerate}
\setcounter{enumi}{\theenumCount}
\item $\ds \frac{d}{dt} (t-1)(t+1)$
\item $\ds \frac{d}{dx} \dfrac{x^4 - 5x^3 + 6x^2}{x^2}$
\setcounter{enumCount}{\theenumi}
\end{enumerate}
\end{multicols}
\vfill

\begin{multicols}{2}
\begin{enumerate}
\setcounter{enumi}{\theenumCount}
\item $\ds \frac{d}{dt} (\sqrt{t})^3$
\item $\ds \frac{d}{dx} \dfrac{x^3}{\sqrt{x}}$
\setcounter{enumCount}{\theenumi}
\end{enumerate}
\end{multicols}
\vfill


\begin{multicols}{2}
\begin{enumerate}
\setcounter{enumi}{\theenumCount}
\item $\ds \frac{d}{dx} x^2 e^x$
\item $\ds \frac{d}{dx} x \ln x$
\setcounter{enumCount}{\theenumi}
\end{enumerate}
\end{multicols}
\vfill



\begin{enumerate}
\setcounter{enumi}{\theenumCount}
\item Suppose that $f(x)$ and $g(x)$ are functions such that $f(3) = 5$, $g(3) = 2$ and $f'(3) = 1$ while $g'(3) = -2$. 
\begin{enumerate}
\item Find the derivative of $f(x) \cdot g(x)$ when $x = 3$. 
\vfill
\item Find the derivative of $f(x) \cdot f(x)$ when $x = 3$. 
\vfill
\end{enumerate}

\newpage
\item A manufacturer's total monthly revenue is $R(x) = 240 x - 0.05 x^2$ where $x$ is the number of products sold. 
\begin{enumerate}
\item Find the marginal revenue $R'(x)$. 
\vfill
\item Calculate $R'(80)$.  
\vfill

\item Calculate $R(81) - R(80)$.  Is it close to the previous answer?  Should it be? 
\vfill

\end{enumerate}


\item Suppose that the total cost to produce $x$ units is $C(x) = 3x^2 + x + 500$.  
\begin{enumerate}
\item Find the marginal cost $C'(x)$. 
\vfill

\item Calculate $C'(40)$.  
\vfill
\end{enumerate}

\item The average cost per item from the previous problem is $A(x) = 3x + 1 + \dfrac{500}{x}$.  
\begin{enumerate}
\item Find the derivative of the average cost function. 
\vfill

\item Is the average cost increasing or decreasing when the level of production is $x=10$?  
\vfill
\end{enumerate}

\end{enumerate}


\vfill

%\item A coffee cup is cooling so that its temperature (in Celsius) is given by $T(t) = 80 e^{-0.4t}$ where $t$ is the time in minutes.   
%\vfill


\end{document}


%\item A fertilizer company estimates that if they produce $x$ tons of fertilizer, their revenue will be $R(x) = \dfrac{2x}{\left( 1+\frac{1}{200}x^2 \right)}$ (in thousands of dollars).  Find the marginal revenue $R'(10)$.  If the company is currently producing $10$ tons of fertilizer, should they increase or decrease production to increase revenue?
%\vfill
%
%
%\item Suppose 6 lbs. of salt is dissolved in a tank containing 50 gallons of water.  If the amount of water in the tank is increasing at 1 gallon per minute, then the concentration of the salt is $C(t) = \dfrac{6}{50+t}$ pounds per gallon.  What is the rate at which the concentration is decreasing after 10 minutes?
%\vfill
%
%
%\setcounter{enumCount}{\theenumi}
%\end{enumerate}
%
%
%\noindent
%In 2021, the Virginia state income tax for individuals is calculated as follows.
%\begin{center}
%\begin{tabular}{l|l}
%\hline
%Taxable Income & Tax Calculation \\ \hline
%0 to \$3,000 & 2\% \\
%\$3,000 to \$5,000  & \$60 $+$ 3\% of excess over \$3,000 \\
%\$5,000 to \$17,000 & \$120 $+$ 5\% of excess over \$5,000 \\
%\$17,000$+$  & \$720 $+$ 5.75\% of excess over \$17,000 \\ \hline
%\end{tabular} 
%\end{center}
%\begin{enumerate}
%\setcounter{enumi}{\theenumCount}
%\item The derivative of $T(x)$ is called the marginal tax rate.  Find the formula for $T(x)$ when $x$ is over \$17,000.  Then find the derivative $T'(x)$ for that tax bracket.  
%\vfill 
%
%\setcounter{enumCount}{\theenumi}
%\end{enumerate}


\end{document}
