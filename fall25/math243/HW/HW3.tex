\documentclass[10pt]{article}
\usepackage[margin=0.4in]{geometry}
\usepackage{amsmath}
\usepackage{enumitem}
\usepackage{multicol}
\usepackage{tikz}
\usetikzlibrary{shapes.geometric}
\usepackage{soul}

\newcommand{\ds}{\displaystyle}
\newcommand{\on}{\operatorname}


\begin{document}
\newcounter{enumCount}
\pagestyle{empty}
\subsection*{Homework 3 - Math 243 \hfill Name: \underline{\hspace*{2in}}}


\noindent
\textit{Solve the following initial value problems using separation of variables.}

\begin{multicols}{2}
\begin{enumerate}
\setcounter{enumi}{\theenumCount}
\item $\dfrac{dy}{dt} + y = 5$, $y(0) = 20$.


\item $\dfrac{dx}{dt} = 3t^2 x + x$, $x(0) = 3$.

\setcounter{enumCount}{\theenumi}
\end{enumerate} 
\end{multicols}
\vfill

\noindent

\begin{enumerate}
\setcounter{enumi}{\theenumCount}
\item Consider the differential equation $\dfrac{dx}{dt} = x - t^2$ with initial value $x(0) = 3$.  Without using a computer, compute the first 3 steps of Euler's method with a step size of $h = 1$.  Enter your results in the table below. 

\renewcommand{\arraystretch}{1.5}
\begin{tabular}{|c|c|}
\hline
~~~~~~ $t$ ~~~~~~ & ~~~~~~ $x$ ~~~~~~ \\ \hline
0 & 3 \\ \hline
1 &   \\ \hline
2 &   \\ \hline
3 &   \\ \hline
\end{tabular}

\bigskip
\bigskip


\item Consider the differential equation $\dfrac{dy}{dt} = 3y - 1$ with initial value $y(0) = 2$.  
\begin{enumerate}
\item Use separation of variables to solve this differential equation.
\vfill


\item Use Euler's method on a computer to estimate $y(2)$ using a step size of $h = 0.5$.  
\bigskip
\bigskip


\item How far apart is the exact value of $y(2)$ and the Euler's method approximation (accurate to 4 decimal places)?
\vspace*{1.0in}
\end{enumerate}


\newpage
\item Suppose that a simple electric circuit has a resistor with resistance $R$ in ohms ($\Omega$), a capacitor with capacitance $C$ in farads (F) and a (time-dependent) voltage source that provides $E(t)$ volts (V).  The voltage drop across the capacitor $E_C$ satisfies the differential equation
$$RC \dfrac{dE_C}{dt} + E_C = E(t).$$
Suppose that  $R = 2~\Omega$, $C = 1$ F, and $E(t) = 5 \sin (2 \pi t)$ volts where $t$ is measured in seconds. If the initial voltage drop across the capacitor is $E_C(0) = 10$ V, then use Euler's method with a step size of $h = 0.1$ seconds to estimate the $E_C(t)$ when $t = 5$.  
\vfill

\item The velocity of an object falling near the surface of the Earth is governed by the drag equation:
$$m \dfrac{dv}{dt} = mg - \tfrac{1}{2} \rho v^2 A C_d$$
where $m$ is the mass of the object, $g$ is the acceleration of gravity, $\rho$ is the density of the fluid in which the object is falling, $A$ is the cross-sectional area of the object, and $C_d$ is the coefficient of drag.  What is the equilibrium solution to this equation? Is it stable?  And what does it mean about the falling object?  
\vfill


\item Use Euler's method with $h = 0.01$ to estimate the velocity of a 1 kg sphere with a radius of $0.1$ meters that has been falling for 10 seconds.  Assume that $g = 9.8 \text{ m}/\text{s}^2$, $C_d = 0.47$, $\rho = 1.2$ kg/m$^3$, and $A = 0.314$ m$^2$. 
\vfill

%\item Draw a rough sketch of a slope field for the following differential equations. 
%\begin{center}
%\begin{tikzpicture}[scale=0.8]
%\draw (-3.5, 4) node[right] {(a) $y' = -xy$};
%\draw[gray!40] (-2.5,-2.5) grid (2.5, 2.5);
%\draw[thick,<->] (-3,0) -- (3,0);
%\draw[thick,<->] (0,-3) -- (0,3);
%
%\begin{scope}[xshift =12cm]
%\draw (-3.5, 4) node[right] {(b) $y' = (x+1)(x-2)$};
%\draw[gray!40] (-2.5,-2.5) grid (2.5, 2.5);
%\draw[thick,<->] (-3,0) -- (3,0);
%\draw[thick,<->] (0,-3) -- (0,3);
%\end{scope}
%\end{tikzpicture}
%\end{center}
%\bigskip

\end{enumerate}


\end{document}
