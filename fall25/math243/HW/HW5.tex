\documentclass[10pt]{article}
\usepackage[margin=0.4in]{geometry}
\usepackage{amsmath}
\usepackage{enumitem}
\usepackage{multicol}
\usepackage{tikz}
\usetikzlibrary{shapes.geometric}
\usepackage{soul}

\newcommand{\ds}{\displaystyle}
\newcommand{\on}{\operatorname}


\begin{document}
\newcounter{enumCount}
\pagestyle{empty}
\subsection*{Homework 5 - Math 243 \hfill Name: \underline{\hspace*{2in}}}


\noindent


\noindent
\textit{Solve the following partially coupled systems analytically.}

\begin{multicols}{2}
\begin{enumerate}
\setcounter{enumi}{\theenumCount}
\item \hl{insert!}


\item \hl{insert}

\setcounter{enumCount}{\theenumi}
\end{enumerate} 
\end{multicols}
\vfill




\begin{enumerate}
\setcounter{enumi}{\theenumCount}

\item In the ocean, cod eat krill and seals eat both cod and krill.  Write a system of three differential equations to model the populations of the krill $K$, the cod $C$, and the seals $S$.  Use lower case letters for any constants you need and you can assume that the krill population would obey a constrained growth model (i.e., a logistic model) in the absence of predators. 
\vfill

%\item In class we looked at a predator-prey system 
%\begin{align*}
%\dfrac{dR}{dt} &= 2R - RF \\
%\dfrac{dF}{dt} &= -5F + RF \\
%\end{align*}
%In this model, the population of rabbits would grow exponentially in the absence of foxes.  What if we changed the model so that they rabbits population would follow a logistic growth model $R' = 2R(1-\frac{R}{10}) - RF$.  Would that change the behavior of the system?  Use a direction field to justify your answer.  \hl{Save this one for after we have classified equilibria?}
% Show that (1,5) is an equilibrium solution. 
% Calculate the Jacobian matrix for the system at (1,5).
% Use the classification rule for equilibria to identify the type of equilibrium at (1,5).
%\vfill

\setcounter{enumCount}{\theenumi}
\end{enumerate}

\textit{Find all equilibrium solutions for the following systems of equations.}

\begin{multicols}{2}
\begin{enumerate}
\setcounter{enumi}{\theenumCount}
\item ~ \\ $\begin{array}{c} \dfrac{dx}{dt} = x - 2y + 4 \\ ~ \\ \dfrac{dy}{dt} = 2x + y - 7 \end{array}$.


\item ~ \\ $\begin{array}{c} \dfrac{dR}{dt} = 5R - 0.5 RF \\ ~ \\ \dfrac{dF}{dt} = -4F + 0.1RF \end{array}$.

\setcounter{enumCount}{\theenumi}
\end{enumerate} 
\end{multicols}
\vfill


\newpage
\begin{enumerate}
\setcounter{enumi}{\theenumCount}
\item Consider the differential equation $\dfrac{dv}{dt} + 6v = e^{2t}$.  
\begin{enumerate}
\item What is the corresponding homogeneous differential equation? 
\vfill

\item What is the general solution of the homogeneous differential equation?
\vfill

\item Find one particular solution for the original equation of the form $v(t) = Ae^{2t}$. Then combine the particular solution and the homogeneous solution to express the general solution.  
\vfill
\end{enumerate}





\item Consider the differential equation $\dfrac{dy}{dt} + y = 6 \cos x.$
\begin{enumerate}
\item Find constants $A$ and $B$ such that $y(t) = A \cos t + B \sin t$ is a solution to this ODE.
\vfill


\item Find the general solution of $\dfrac{dy}{dt} + y = 6 \cos x$. 
\vfill
\end{enumerate}

\setcounter{enumCount}{\theenumi}
\end{enumerate}


\end{document}
