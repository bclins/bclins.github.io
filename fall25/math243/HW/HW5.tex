\documentclass[10pt]{article}
\usepackage[margin=0.4in]{geometry}
\usepackage{amsmath}
\usepackage{enumitem}
\usepackage{multicol}
\usepackage{tikz}
\usetikzlibrary{shapes.geometric}
\usepackage{soul}

\newcommand{\ds}{\displaystyle}
\newcommand{\on}{\operatorname}


\begin{document}
\newcounter{enumCount}
\pagestyle{empty}
\subsection*{Homework 5 - Math 243 \hfill Name: \underline{\hspace*{2in}}}


\noindent


\noindent
\textit{Solve the following partially coupled systems analytically.}

\begin{multicols}{2}
\begin{enumerate}
\setcounter{enumi}{\theenumCount}
\item ~ \\ $\begin{array}{rcl} \dfrac{dx}{dt} & = & -4 x \\ ~ & & \\ \dfrac{dy}{dt} & = & 3x + 2y  \end{array}$


\item ~ \\ $\begin{array}{rcl} \dfrac{dx}{dt} & = & x y \\ ~ & & \\ \dfrac{dy}{dt} & = & y + 1  \end{array}$

\setcounter{enumCount}{\theenumi}
\end{enumerate} 
\end{multicols}
\vfill




\begin{enumerate}
\setcounter{enumi}{\theenumCount}

\item In the ocean, cod eat krill and seals eat both cod and krill.  Write a system of three differential equations to model the populations of the krill $K$, the cod $C$, and the seals $S$.  Use lower case letters for any constants you need and you can assume that the krill population would obey a constrained growth model (i.e., a logistic model) in the absence of predators. 
\vfill

%\item In class we looked at a predator-prey system 
%\begin{align*}
%\dfrac{dR}{dt} &= 2R - RF \\
%\dfrac{dF}{dt} &= -5F + RF \\
%\end{align*}
%In this model, the population of rabbits would grow exponentially in the absence of foxes.  What if we changed the model so that they rabbits population would follow a logistic growth model $R' = 2R(1-\frac{R}{10}) - RF$.  Would that change the behavior of the system?  Use a direction field to justify your answer.  \hl{Save this one for after we have classified equilibria?}
% Show that (1,5) is an equilibrium solution. 
% Calculate the Jacobian matrix for the system at (1,5).
% Use the classification rule for equilibria to identify the type of equilibrium at (1,5).
%\vfill

\item Suppose that $\mathbf{F}(1,2) = (2,3)$.  If you apply Euler's method to the system of differential equations 
$$\dfrac{d\mathbf{Y}}{dt} = \mathbf{F}(\mathbf{Y})$$
with initial condition $\mathbf{Y}_0 = (1,2)$,
then what is the value of $\mathbf{Y}$ after one step with $h = 0.1$? 
\vfill

\newpage
\item Write the following 2nd order differential equation as a system of first order differential equations.  You do not need to solve it.
$$2y'' - 5ty' + \sin y = 0.$$
\vfill
% v = y' 
% v' = 5tv - \sin y
%\item The motion of a pendulum obeys the differential equation $\dfrac{d^2 \theta}{dt^2} = - \dfrac{g}{L} \sin \theta.$ 
%Convert this second order equation into a system of two first order equations. 
%\vfill

\setcounter{enumCount}{\theenumi}
\end{enumerate}



\begin{enumerate}
\setcounter{enumi}{\theenumCount}
\item The \textbf{Van der Pol equation} is 
$$\dfrac{d^2 x}{dt} - (1-x^2) \dfrac{dx}{dt} + x = 0.$$
We can study this equation numerically by converting to the system of equations
\begin{align*}
\dfrac{dx}{dt} &= v \\ 
\dfrac{dv}{dt} &= (1-x^2)v - x.
\end{align*}
Use Euler's method to approximate the solution of this equation with initial condition $(x_0,v_0) = (1,1)$, and a step size of $h = 0.01$ after $N = 1{,}000$ steps.  What do you get for $x(10)$ and $v(10)$?  
\vfill




\item Use Euler's method with $h = 0.01$ and $N = 300$ steps to approximate the values of $x(3)$ and $y(3)$ if $(x_0, y_0) = (0.5,-2)$ and 
\begin{align*}
\dfrac{dx}{dt} &=  y \\ 
\dfrac{dy}{dt} &= 2x + 3y^2.
\end{align*}
\vfill

\end{enumerate}

\end{document}
