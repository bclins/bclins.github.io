\documentclass[11pt]{article}
\usepackage[margin=0.75in]{geometry}
\usepackage{amsmath}
\usepackage{enumitem}
\usepackage{color,soul}

\usepackage{multicol}


\begin{document}
\newcounter{enumCount}
\pagestyle{empty}
\subsection*{Math 105 - Homework 2 \hfill Name: \underline{\hspace*{2in}}}
\textit{Solve each equation. Show your work. No calculators.} 

\begin{multicols}{3}
\begin{enumerate}
\item $x+5 = 7$
\item $4 - x = 3$
\item $-3x = 4$
\setcounter{enumCount}{\theenumi}
\end{enumerate}
\end{multicols}
\vfill

\begin{multicols}{3}
\begin{enumerate}
\setcounter{enumi}{\theenumCount}
\item $\dfrac{2}{3}u = \dfrac{1}{5}$
\item $\dfrac{y}{5} = \dfrac{3}{4}$
\item $4r - 3 = 13$
\setcounter{enumCount}{\theenumi}
\end{enumerate}
\end{multicols}
\vfill

\begin{multicols}{3}
\begin{enumerate}
\setcounter{enumi}{\theenumCount}
\item $2x + 5 = 11$
\item $5-y = \dfrac{3}{10}$
\item $4 - 2s =  8s - 10$
\setcounter{enumCount}{\theenumi}
\end{enumerate}
\end{multicols}
\vfill

\begin{multicols}{3}
\begin{enumerate}
\setcounter{enumi}{\theenumCount}
\item $3 + 2y = 4 - 3y$
\item $\dfrac{t}{4} + 2 = \dfrac{t}{3}$
\item $\dfrac{x+4}{120} = \dfrac{1}{30}$
\setcounter{enumCount}{\theenumi}
\end{enumerate}
\end{multicols}
\vfill


\begin{multicols}{3}
\begin{enumerate}
\setcounter{enumi}{\theenumCount}
\item $\dfrac{3}{10}r + 7= 4$
\item $\dfrac{-1}{2}x - 3 = \dfrac{3}{2}$
\item $\dfrac{2}{x} = \dfrac{4}{x+3}$
\setcounter{enumCount}{\theenumi}
\end{enumerate}
\end{multicols}
\vfill

\begin{multicols}{3}
\begin{enumerate}
\setcounter{enumi}{\theenumCount}
\item $3(x-5) = 12$
\item $\dfrac{x-2}{5} = 10$
\item $\dfrac{x+5}{x} = 2$
\setcounter{enumCount}{\theenumi}
\end{enumerate}
\end{multicols}
\vfill

\begin{multicols}{3}
\begin{enumerate}
\setcounter{enumi}{\theenumCount}
\item $5-\dfrac{u}{7} = 3$
\item $\dfrac{2}{q} -3 =  4- \dfrac{5}{q}$
\item $3z-5 + z = 7 z - 2 - 3z$

\setcounter{enumCount}{\theenumi}
\end{enumerate}
\end{multicols}
\vfill

\begin{multicols}{3}
\begin{enumerate}
\setcounter{enumi}{\theenumCount}
\item $\dfrac{4x-(-x)+3x}{2} = -8$
\item $(5x+4)+(2x+1) = x+5$
\item $\dfrac{x}{12}+\dfrac{x}{-3} = 10$
\setcounter{enumCount}{\theenumi}
\end{enumerate}
\end{multicols}
\vfill



\hfill More $\longrightarrow$

\newpage


\begin{multicols}{3}
\begin{enumerate}
\setcounter{enumi}{\theenumCount}
\item $\dfrac{4}{2-x} = -1$
\item $\dfrac{u}{u+4} = 3$
\item $\dfrac{x+1}{x+3}+\dfrac{x+2}{x+3} = 5$
\setcounter{enumCount}{\theenumi}
\end{enumerate}
\end{multicols}
\vfill


\begin{multicols}{3}
\begin{enumerate}
\setcounter{enumi}{\theenumCount}
\item ${r^2+4} = 20$
\item $\dfrac{x^3 - 7}{2} = 10$
\item $\sqrt{y^2 - 5} = 2$
\setcounter{enumCount}{\theenumi}
\end{enumerate}
\end{multicols}
\vfill

\noindent
\textit{Solve each of the following equations for the indicated variable.}
\begin{multicols}{2}
\begin{enumerate}
\setcounter{enumi}{\theenumCount}
\item $E = mc^2$, solve for $m$ 
\item $16 = b^2 - 4ac$, solve for $a$
\setcounter{enumCount}{\theenumi}
\end{enumerate}
\end{multicols}
\vfill

\begin{multicols}{2}
\begin{enumerate}
\setcounter{enumi}{\theenumCount}
\item $A = \tfrac{1}{2} bh$, solve for $h$
\item $P = 2\ell + 2w$, solve for $\ell$
\setcounter{enumCount}{\theenumi}
\end{enumerate}
\end{multicols}
\vfill

\begin{multicols}{2}
\begin{enumerate}
\setcounter{enumi}{\theenumCount}
\item $V = \tfrac{1}{3} \pi r^2 h$, solve for $h$.
\item $S = 2\pi r^2 + 2 \pi r h$, solve for $h$. 
\setcounter{enumCount}{\theenumi}
\end{enumerate}
\end{multicols}
\vfill


%\noindent
%\textit{Find the x-value where the two lines cross.}
%\begin{multicols}{3}
%\begin{enumerate}
%\setcounter{enumi}{\theenumCount}
%\item $y = x$ and $y=2x+1$
%\item $y = -2x + 3$ and $y = -3$
%\item $x+y = 1$  and $x = y$
%\setcounter{enumCount}{\theenumi}
%\end{enumerate}
%\end{multicols}
%\vfill
%
%\noindent
%\textit{Draw a graph, and find an equation for the line that meets the following criteria.}
%\begin{multicols}{2}
%\begin{enumerate}
%\setcounter{enumi}{\theenumCount}
%\item Slope is 2 and hits the x-axis at $x = -3$.
%\item Passes through the points $(2,1)$ and $(0,3)$.
%\setcounter{enumCount}{\theenumi}
%\end{enumerate}
%\end{multicols}
%\vfill
%\vfill

\vfill

\end{document}
