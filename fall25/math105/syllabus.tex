\documentclass[10pt]{article}

\usepackage[empty]{fullpage}
\usepackage{hyperref}
\usepackage{tikz}
\usepackage{soul}

\begin{document}
\noindent
%
% REQUIRED SYLLABUS CONTENT AS OF SPRING 2025:
% 1. Course Information
%     - Course prefix and number, section number, title, and credit hours
%     - Course description (include full Academic Catalogue description with prerequisites)
%     - Semester and year
%     - Class meeting days, time, and location
%     - Faculty member’s name, contact information, and office hours
%     - Required course materials and texts
%     - Assignment descriptions and calculation of grades
%     - Course schedule
%     - Policies governing late work, make-up assignments, and attendance
% 2. Student Learning Information
%     - Course’s relationship to degree program
%     - Student learning outcomes for major/standalone minor
%     - Course learning outcomes (see template)
% 3. All syllabi must contain the following institutional policies. The wording for these policies should be copied and pasted from the syllabus template.
%     - Grading Scale
%     - Honor Code
%     - Policies for use of AI on assignments
%     - Accommodations


\begin{center}
\includegraphics[scale=0.6]{../../../HSC.png} 
\bigskip

\textbf{Math 105: Prep for Calculus (1 credit)} \\
Fall 2025
\end{center}

\noindent
\begin{tabular}{|l|l|}
\hline
Instructor & Brian Lins \\ \hline
Email Address & blins@hsc.edu \\ \hline
Course Meeting Time & M 2:30 - 3:20pm \\ \hline
Course Meeting Location & Pauley 222 \\ \hline
Office Hours & WF 10:30 - 11:30am \& W 2:30 - 4:00pm \\ 
& See the course website: \url{https://bclins.github.io} ~ \\ \hline
Textbook & OpenIntro Statistics, 4th edition by Diez,\\ &  Cetinkaya-Rundel, \& Barr \\ \hline
\end{tabular}


\subsubsection*{Course Description}

A course designed to maximize students’ potential to succeed in calculus by reinforcing basic mathematical skills. Specific topics include functions and their graphs, algebra, and trigonometry.

\subsubsection*{Course Learning Objectives}

\begin{itemize}

\item Students will improve their facility with algebraic manipulations.
\item Students will review \& master the algebra needed to succeed in calculus.

\end{itemize}

\subsubsection*{Required Materials}

None. See the course website for links to the free textbook.

%\subsubsection*{Assignment Descriptions and Grade}
%
\subsubsection*{Attendance Policy}

Attendance in this class is required. We will only meet 14 times all semester, so missing even one class is a big deal. Repeated absences may result in a forced withdrawal from the course. You are responsible for any material you miss due to absence. Please let me know ahead of time if you know that you will not be able to attend class.

\subsubsection*{Grading Policy}

The term grade will be based on the following factors.

\begin{center}
\begin{tabular}{|l|c|}
\hline
Component      & Proportion \\ \hline
Homework & 60\% \\
Midterm exam  & 20\% \\
Final exam  & 20\% \\ \hline
\end{tabular}
\end{center}

\subsubsection*{In-Class Problems \& Homework}
  
There is a saying that, ``you learn math by doing math." This is very true! During most class periods you will be asked to solve problems in-class. Any problems you do not complete during class become homework problems. Homework problems are due at the beginning of the next class.  




\subsubsection*{Exams}

There will be a midterm and final exam.  After the exams, you will be able to submit test corrections to get 50\% of points lost back.  


\subsubsection*{Late Work and Make-Up Assignment Policy}

Please let me know in advance if you will be missing class. If you let me know ahead of time that you will be missing class for a school sponsored event, then we can plan an alternative assignment. If you don't let me know until after the fact, then it will be too late.  

\subsubsection*{Grading Scale} 

This course adheres to the grades and quality points described in the \href{https://www.hsc.edu/academic-catalogues}{Academic Catalogue}. Consult the Academic Catalogue for a detailed description. 


\subsubsection*{Honor Code}

Students are expected to abide by the Honor Code for all assignments unless a professor indicates otherwise. Students should consult the Academic Catalogue and The Key: The Hampden-Sydney College Student Handbook for the College’s description of the Honor Code and what it identifies as infractions of the Honor Code.

\subsubsection*{Artificial Intelligence Policy}

Artificial intelligence (AI) generators and large language models (LLMs) often rely on existing published materials, and copying or paraphrasing materials generated by AI without attribution is plagiarism. Professors may permit students to use AI generators or LLMs in a variety of ways in their own classes. Those students, however, must not assume that those policies transfer to other classes.

\subsubsection*{Accommodations}

Hampden-Sydney College is committed to ensuring equitable access to its education programs for all students. Under the administration of the Department of Culture and Community, the Office of Accessibility Services (OAS) coordinates reasonable accommodations for qualified students with disabilities. If you wish to seek accommodations for this class, please contact Dr. Melissa Wood, Director of Title IX, Access, and Student Advocacy, at 434-223-6061 or at \url{mwood@hsc.edu}. Additional information may be found here: \url{https://www.hsc.edu/academics/academic-services/disability-services}. Appropriate documentation of disability will be required. For students who have an accommodations letter from OAS, it is essential that you correspond with your professor as soon as possible to discuss your accommodation needs for the course so that appropriate arrangements may be made. 
\end{document}


