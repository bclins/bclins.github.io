\documentclass[12pt]{exam}
\usepackage[empty]{fullpage}
\usepackage{amsmath}
\usepackage{tikz}
\usepackage{hyperref}

\newcommand{\blank}[1]{\underline{\hspace*{#1}}}
\newcommand{\ds}{\displaystyle}
\newcommand{\mymod}{\, \mathrm{mod}\,}

\begin{document}

\subsubsection*{Project 4 \hfill Math 121}

\textit{Please type your solutions to the questions below in Microsoft Word or Google Docs and turn in your solutions in class on \textbf{Monday, Dec 8}. Use complete sentences to write your answers. It is okay to discuss the problems with other students, but all of your solutions must be explained in your own words. } \\

\noindent
Is there a relation between education level and religious observance in the United States? Below is data from a sample of 2{,}002 American adults who participated in the General Social Survey (GSS).
One of the questions asked on the GSS is ``Have you attended religious services in the last week?"  Here are the results broken down by education level for adults surveyed with at least a high school education. 

\begin{center}
\begin{tabular}{l|cccc} 
& \multicolumn{4}{c}{Highest Degree Held} \\ \hline 
 & High School & 2-Year College & 4-Year College & Graduate \\ \hline
Attended services & 400 & 62 & 146 & 76 \\
Did not attend services & 880 & 101 & 232 & 105 \\ \hline
Total & 1280 & 163 & 378 & 181 \\ 
\end{tabular}
\end{center}

\begin{questions}
\item Make a stacked bar graph showing the \textbf{column percentages} in this two way table. You can use Excel to make the graph. What is the biggest difference visible between the different groups?
\begin{solution}
Here are the percentages: 
\begin{center}
\begin{tabular}{l|cccc}
 & HS & 2-Year & 4-Year & Grad \\ \hline
Attendance & 31.3\% & 38.0\% & 38.6\% & 42.0\% \\
\end{tabular}
\end{center}
\end{solution}
\vfill

\item How many degrees of freedom does the table above have for the chi-squared test?
\begin{solution}
3
\end{solution}
\vfill

\item For this table, $\chi^2 = 14.2$. What is the corresponding p-value? What does it mean about education level versus attendance at religious services? 
\begin{solution}
$p= 0.265\%$.  This means that there is strong evidence that education level is associated with attendance at religious services.  It looks like more educated people are significantly more likely to go to church.  
\end{solution}
\vfill

\item Which group is the least likely to have attended religious services?   
\begin{solution}
People with just a high school degree.
\end{solution}
\vfill

\item If you remove that group, then for the remaining three groups, $\chi^2 = 0.73$ which has a p-value of $69.4\%$.  What do these three groups have in common, and what does the chi-squared test say about the association between education level and religious attendance when you only look at these three groups? 
\begin{solution}
The other three groups all have some kind of higher education, and the differences between which type of higher degree and religious attendance are not statistically significant. 
\end{solution}
\vfill

\end{questions}
\end{document}
